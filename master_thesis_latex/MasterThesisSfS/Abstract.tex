\chapter*{Abstract}

Conventional time series analysis methods typically assume observations at
evenly spaced time intervals.
In response to the challenges posed by irregularly sampled time series data and
motivated by a real-world blood pressure time series example,
this study introduces Gaussian processes as a powerful tool for analysis.

We first highlight the limitations of conventional approaches when dealing with
irregularly spaced data.
Subsequently, we demonstrate how Gaussian process regression overcomes these
limitations by enabling the modeling of time series in continuous time.

Drawing inspiration from the real-world example, we conduct a simulation study
to assess the suitability of Gaussian process regression for estimating blood
pressure values from irregularly spaced simulated measurements.
The measurements were simulated through the simulation of a blood pressure time
series that mimics real-world dynamics by
featuring cyclic, autoregressive, and long-term trend components.

We evaluate the performance of estimating clinically relevant target measures,
including time in the target range, one-week, one-day, and one-hour mean blood pressure.
Performance assessment involves calculating confidence interval coverage and width
through repeated simulations.

We compare the estimation performance of Gaussian process regression with
baseline methods, among them linear and spline regression,
considering varying data densities and sampling patterns.
These patterns include uniform and seasonal sampling, with the latter reflecting circadian rhythms.

Results reveal that Gaussian process regression consistently outperforms baseline methods
across all target measures, data densities, and sampling patterns.
Its superiority becomes more evident with increased data density and seasonal sampling,
particularly for smaller time windows like the one-hour or one-day mean blood pressure.

Linear regression exhibits its strengths in estimating the one-week mean even
under low data density conditions, but its improvement saturates with additional data.
Spline regression performs well with abundant uniformly sampled data but struggles
with seasonal sampling, where performance deteriorates as more data is added.


%\chapter*{Abstract}
%Conventional methods for time series analysis generally assume
%observations at equally spaced time points.
%Motivated by a real-world example of a blood pressure time series observed at irregularly
%spaced time points, this study aims at presenting Gaussian processes for the analysis of
%irregularly spaced time series.
%
%We firs present the limitation of a conventional method of dealing with
%irregularly spaced data. Further we show how Gaussian process regression
%can overcome these limitations, since it can model a time series in continuous time.
%
%Inspired by the real-world example a simulation study is performed that
%should show the suitability of Gaussian process regression for the purpose
%of estimating blood pressure values from irregularly spaced measurements.
%First blood pressure time series and its observations are simulated such
%that they would closely mimic the real-world blood pressure.
%Based on
%the real-world blood pressure data, the
%simulated time blood pressure time series is defined to feauture
%a cyclic component that mimics the circadian cycle, an autoregressive component
%and a long term trend component.
%Suitability is assessed by assessing the performance of estimating the expected value
%and confidence intervals of some
%clinically relevant target measures from one-week of blood pressure measurements.
%The target measures are time in target range as well as the
%one-week, one-day and one-hour mean blood pressure.
%Performance is assessed by calculating confidence interval coverage and width
%from repeated simulation of a true blood pressure
%time series.
%The estimation performance is compared to those of baseline methods, spline regressin
%and linear regression. The baseline method can work on irregularly spaced
%samples but do not  explicitly model
%the time series component. Different degree of data density and sampling patterns have
%been investigated. The uniform sampling pattern produces observations uniformly
%sampled across time.
%The seasonal sampling pattern would follow the circadian cycle, simulating
%higher data density during the day than during the night.
%
%Gaussian process regression outperforms the baseline methods considering all
%target measures, data densities and sampling patterns. The superiority
%is more pronounced when data density increases, when there is seasonal sampling
%and when considering smaller time windows such as the one-hour
%or one-day mean blood pressure. Linear regression generally produces the narrowest
%confidence interval can adequately estimate
%the one-week mean even under very low data density, while not imporoving much with
%more data. Spline regression performs well when there is plenty of data and
%unifrom sampling. Spline and linear regression both perfrom worse
%in the face of seasonal sampling and performance decrases the more data is added in this
%scenario.






%%% Local Variables: 
%%% mode: latex
%%% TeX-master: "MasterThesisSfS"
%%% End: 
