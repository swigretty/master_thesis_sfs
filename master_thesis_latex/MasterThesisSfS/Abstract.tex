\chapter*{Abstract}
Conventional time series analysis methods often assume evenly spaced observations,
which may not always reflect real-world data collection constraints.
Drawing inspiration from real-world scenarios, this study highlights Gaussian
processes as a potent tool for analyzing irregularly sampled
time series data.

Using a simulated blood pressure dataset designed to mimic real-world dynamics,
including cyclic, autoregressive, and long-term trend components,
we evaluate Gaussian process regression's performance in estimating
blood pressure values from one week of irregularly spaced measurements.
We assess the accuracy of credible interval estimation for clinically relevant
target measures through repeated simulations, comparing it with baseline methods,
such as spline and linear regression, accompanied by bootstrapped confidence intervals.
Our investigation extends to the impact of varying data density and sampling patterns,
specifically comparing uniform and seasonal sampling, where data density fluctuates with the circadian cycle.
Results consistently demonstrate Gaussian process regression's superior
performance across all target measures,
data densities, and sampling patterns.
While linear regression, featuring a linear trend and sinusoidal component,
serves as a viable baseline under low-data scenarios, it exhibits notable
estimation bias due to its inherent constraints, and thus does not improve with more data.
In contrast, spline regression offers flexibility but falters with seasonal
sampling due to its lack of prior function knowledge.
Gaussian process regression strikes a balance between flexibility and encoding
prior beliefs about the true blood pressure function, yielding accurate results
even with sparse, seasonally sampled data.
Notably, it explicitly models the autoregressive component,
yielding more precise credible intervals compared to the bootstrapped
confidence intervals of the baseline methods.
In summary, this study shows the potential of Gaussian processes as a
robust tool for the analysis of irregularly sampled time series data,
exemplified through the lens of blood pressure estimation.

%
%We first highlight the limitations of conventional approaches when dealing with
%irregularly spaced data.
%Subsequently, we demonstrate how Gaussian process regression overcomes these
%limitations by enabling the modeling of time series in continuous time.
%
%particularly for smaller time windows like the one-hour or one-day mean blood pressure.
%
%While linear regression is  exhibits its strengths in estimating the one-week mean even
%under low data density conditions, but its improvement saturates with additional data.
%Spline regression performs well with abundant uniformly sampled data but struggles
%with seasonal sampling, since it
%
%
%
%
%We evaluate the performance of estimating clinically relevant target measures,
%including time in the target range, one-week, one-day, and one-hour mean blood pressure.
%Performance assessment involves calculating confidence interval coverage and width
%through repeated simulations.
%
%We compare the estimation performance of Gaussian process regression with
%baseline methods, among them linear and spline regression,
%considering varying data densities and sampling patterns.
%These patterns include uniform and seasonal sampling. For the latter, data
%density follows circadian rhythms and is higher during daytime than during nighttime.
%
%Results reveal that Gaussian process regression consistently outperforms baseline methods
%across all target measures, data densities, and sampling patterns.
%Its superiority becomes more evident with increased data density and seasonal sampling,
%particularly for smaller time windows like the one-hour or one-day mean blood pressure.
%
%While linear regression is  exhibits its strengths in estimating the one-week mean even
%under low data density conditions, but its improvement saturates with additional data.
%Spline regression performs well with abundant uniformly sampled data but struggles
%with seasonal sampling, since it


%\chapter*{Abstract}
%Conventional methods for time series analysis generally assume
%observations at equally spaced time points.
%Motivated by a real-world example of a blood pressure time series observed at irregularly
%spaced time points, this study aims at presenting Gaussian processes for the analysis of
%irregularly spaced time series.
%
%We firs present the limitation of a conventional method of dealing with
%irregularly spaced data. Further we show how Gaussian process regression
%can overcome these limitations, since it can model a time series in continuous time.
%
%Inspired by the real-world example a simulation study is performed that
%should show the suitability of Gaussian process regression for the purpose
%of estimating blood pressure values from irregularly spaced measurements.
%First blood pressure time series and its observations are simulated such
%that they would closely mimic the real-world blood pressure.
%Based on
%the real-world blood pressure data, the
%simulated time blood pressure time series is defined to feauture
%a cyclic component that mimics the circadian cycle, an autoregressive component
%and a long term trend component.
%Suitability is assessed by assessing the performance of estimating the expected value
%and confidence intervals of some
%clinically relevant target measures from one-week of blood pressure measurements.
%The target measures are time in target range as well as the
%one-week, one-day and one-hour mean blood pressure.
%Performance is assessed by calculating confidence interval coverage and width
%from repeated simulation of a true blood pressure
%time series.
%The estimation performance is compared to those of baseline methods, spline regressin
%and linear regression. The baseline method can work on irregularly spaced
%samples but do not  explicitly model
%the time series component. Different degree of data density and sampling patterns have
%been investigated. The uniform sampling pattern produces observations uniformly
%sampled across time.
%The seasonal sampling pattern would follow the circadian cycle, simulating
%higher data density during the day than during the night.
%
%Gaussian process regression outperforms the baseline methods considering all
%target measures, data densities and sampling patterns. The superiority
%is more pronounced when data density increases, when there is seasonal sampling
%and when considering smaller time windows such as the one-hour
%or one-day mean blood pressure. Linear regression generally produces the narrowest
%confidence interval can adequately estimate
%the one-week mean even under very low data density, while not imporoving much with
%more data. Spline regression performs well when there is plenty of data and
%unifrom sampling. Spline and linear regression both perfrom worse
%in the face of seasonal sampling and performance decrases the more data is added in this
%scenario.






%%% Local Variables: 
%%% mode: latex
%%% TeX-master: "MasterThesisSfS"
%%% End: 
