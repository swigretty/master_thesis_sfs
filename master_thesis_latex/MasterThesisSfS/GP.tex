
\chapter{Gaussian Process}

Normal distribution over functions, i.e. an infinite set of random
variables $(F_t: t \in T_0$ with $T_0=[0, T]$.
The set of random variable is completely defined by the
mean function $m: T_0 \to \mathbb{R}$ and the covariance or kernel
function $k: T_0 \times T_0 \to \mathbb{R}$.
For each $t \in T_0$ there is a $F_t$ such that
$A \subset T_0, A={t_0, \dots t_n}$
it olds that
\[F_A = [F_{t_0}, \dots F_{t_n}] \sim \mathcal{N}(m_A,\,K_{AA})\]
for
\begin{gather*}
    K_{AA} =
    \begin{bmatrix}
        k(t_1, t_1) & k(t_1, t_2) & \dots & k(t_1, t_n)\\
        \vdots  &  & \vdots  & \vdots \\
        k(t_n, t_1)  & k(t_n, t_1) & \dots  & k(t_n, t_n)
    \end{bmatrix} \text{and }
    \mu_A =
    \begin{bmatrix}
        m(t_1) \\
        \vdots \\
        \mu(t_n)
    \end{bmatrix}
\end{gather*}
where $m$ is called the mean function and k is called covariance (kernel) function, which encodes some assumption
about correlation of the response variable at different time points.

The finite marginals $F_{t_1}, \dots, F_{t_n}$ of GP are multivariate Gaussians.

Parametrized by covariance function $k(x,x') = COV(f(x), f(x'))$



