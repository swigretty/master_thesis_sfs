\chapter{Gaussian Process Regression}\label{ch:gaussian-process-regression}
The objective of regression is generally to establish a mapping between the input variable $x$ and
its corresponding output $f(x)$.
In order to solve such a problem one usually needs some additional constraints on $f(x)$.
In chapter \ref{ch:time-series-decomposition-and-regression} we restricted ourselves to the class of linear functions.
However, an alternative approach is to assign prior probabilities to all possible functions,
giving higher probabilities to those considered more plausible. In this Bayesian framework,
inference revolves around the posterior distribution of these functions,
given some potentially noisy observations of $f(x)$.

This chapter begins by providing a formal definition of a Gaussian Process and subsequently explores its application
in solving regression problems.
The arguments presented in this chapter are based on the textbook of \citeauthor{rasmussen_gaussian_2006}.

\section{Gaussian Process Definition}\label{sec:gaussian-process-definition}

A Gaussian process (GP) can be viewed as a gaussian distribution over functions or as an infinite set of random
variables representing the values of the function $f(x)$ at location $x$.
The Gaussian process is thus a generalization of the Gaussian distribution and a formal definition is given
by \citeauthor{rasmussen_gaussian_2006} :

\begin{definition}[Gaussian Process]\label{def:GP}
 A Gaussian process is a collection of random variables, any finite number of which have a joint Gaussian distribution.
\end{definition}


As a (multivariate) Gaussian distribution is defined by its mean and covariance matrix, a GP is
uniquely identified by its mean $m(x)$ and covariance (kernel) function $k(x,x')$.

We write

\begin{gather*}
    f(x) \sim GP(m(x), k(x,x'))
\end{gather*}
with
\begin{gather*}
    m(x) = \mathbb{E}[f(x)] \\
    k(x,x') = \mathbb{E}[(f(x)-m(x))(f(x')-m(x'))]
\end{gather*}

If we assume $X$ to be the index set or set of possible inputs of $f$, then there is a random variable
$F_x := f(x)$ such that for a set $A \subset X$ with $A={x_1, \dots x_n}$ it holds that:

\[F_A = [F_{x_1}, \dots , F_{x_n}] \sim \N(\mu_A,\,K_{AA})\]
for
\begin{gather}\label{def:Kernel-Matrix}
    K_{AA} =
    \begin{bmatrix}
        k(x_1, x_1) & k(x_1, x_2) & \dots & k(x_1, x_n)\\
        \vdots  &  & \vdots  & \vdots \\
        k(x_n, x_1)  & k(x_n, x_1) & \dots  & k(x_n, x_n)
    \end{bmatrix} \text{and }
    \mu_A =
    \begin{bmatrix}
        \mu(x_1) \\
        \vdots \\
        \mu(x_n)
    \end{bmatrix}
\end{gather}

The finite marginals $F_{x_1}, \dots, F_{x_n}$ of the GP thus have a multivariate gaussian distribution.
In our running example we might consider $X$ to be the time interval $T_0=[0, T]$ however it could be higher dimensional.

Note that a GP with finite index set and hence with joint gaussian distribution is just a specific case
of GP. If we assume an ARMA process with gaussian innovations for the blood pressure time series, one can view the time series
as a collection of multivariate normally distributed random variables and thus as a GP.


If we consider the linear regression case from chapter \ref{ch:time-series-decomposition-and-regression} and assume a
prior distribution
on $\beta$, i.e. $\beta \sim N(0, I)$ then the predictive distribution over $\mu = X \beta$ is Gaussian:
\[
    \mu \sim \N(0, XX^{\top})
\]
This is equivalent to a GP with mean function $m(x) = 0$ and kernel function $k(x, x') = x^{\top}x'$.
This special case of gaussian process regression with this specific kernel function is known as Bayesian linear regression
and will be presented in the next section.


\section{Bayesian Linear Regression}\label{subsec:bayesian-linear-regression}
Predictions in the Bayesian regression setting is finding the posterior distribution of
$f^{\ast} := f(x^{\ast})$ at some input $x^{\ast}$, given some potentially noisy observations of $f(x)$.
This is made possible by employing a prior distribution
over the function $f(x)$.
As shown in section \ref{sec:gaussian-process-definition}, a GP is essentially
assuming a Gaussian distribution over functions.
This section however still stays in the domain of parametric models,
in which case we assume a distribution over the parameters of the function $f(x)$,
rather than over the function itself.
In Bayesian linear regression we are thus assuming a distribution over the
regression coefficients $\beta$.

Recall the linear regression model from chapter \ref{ch:time-series-decomposition-and-regression}.
However, we are assuming a more general setting, where the data generating process does not need to be a time series process.
The function is denoted with $f(x)$ instead of $\mu(t)$ and $Y_i$ is again a noisy observation of
$f(x_i)$, where the additive error $E_i$ does not necessarily need to be from a time series process $(E_t: t \in \{t_1, t_2, \dots  t_n \})$.
We obtain the following data generating model:
\begin{align*}
    f(x_i) &= x_i^{\top}\beta, & Y_i &= f(x_i) + E_i,  & (i = 1, \dots n)
\end{align*}
with $x_i \in \mathbb{R}^p$ being again the input vector and $\beta \in \mathbb{R}^p$ is the vector with
the regression coefficients.

In matrix from:
\begin{align*}
    \mathbf{Y} = X \beta + \mathbf{E}
\end{align*}
Where $\mathbf{Y} = [Y_{1}, \dots Y_{n}]^{\top}$ is the observed data,
$X = [x_{1}, \dots x_{n}]^{\top} \in \mathbb{R}^{n \times p}$ is the design matrix.
We assume again gaussian but potentially correlated errors $\mathbf{E} = [E_{1}, \dots E_{n}]^{\top}$:
\begin{gather*}
    \mathbf{E} \sim \N(0, \Sigma_e)
\end{gather*}
If $\mathbf{E}$ is an ARMA process, then every element of the time series $E_{i}$
is itself a sum of innovations.
Therefore, $\mathbf{E}$ is gaussian as long as it has gaussian innovations.

The likelihood, i.e. the probability of the observations $\mathbf{Y}$ given $X$ and $\beta$ is then:

\begin{gather*}
    p(\mathbf{Y}|X,\beta)
    = \frac{1}{(2\pi)^{n/2} \sqrt {\det(\Sigma_e)}}
    \exp(-\frac{1}{2}(y - X\beta)^{\top} \Sigma_e^{-1}(y-X\beta))
    = \N(X \beta, \Sigma_e)
\end{gather*}

Until now the regression model is exactly the same as in chapter \ref{ch:time-series-decomposition-and-regression}.
The Bayesian approach is different in that we additionally assume a prior distribution over the
regression coefficients $\beta$, based on what we believe are likely values for the coefficients.
To stay in the realm of gaussian processes the prior has to be Gaussian and we choose:

\begin{gather*}
    p(\beta) = \N(0, \Sigma_p)
\end{gather*}
Note how the function $f(x_i)=x_i^{\top}\beta$ is now no longer deterministic but a random function.

Given our observations $\mathbf{Y}$  we can use Bayes' theorem to calculate the posterior distribution over $\beta$:
\begin{gather*}
    p(\beta| \mathbf{Y}, X) = \frac{p(\mathbf{Y},\beta|X)}{p(\mathbf{Y}|X)} =
    \frac{p(\mathbf{Y}|X,\beta)p(\beta)}{p(\mathbf{Y}|X)}
\end{gather*}

One approach is to just plug in the expressions for
$p(\mathbf{Y}|X,\beta)$ and $p(\beta|\mathbf{Y}, X)$ from above, with the marginal likelihood:

\begin{gather}\label{eq:marginal-likelihood}
    p(\mathbf{Y}|X) = \int p(\mathbf{Y}|X,\beta) p(\beta) d\beta = \N(0, X \Sigma_p X^{\top} + \Sigma_e)
\end{gather}

The term marginal likelihood arises from the marginalization over the parameter values $\beta$.

Or it can be helpful to combine the coefficients and the observations into a single random vector with
multivariate normal distribution:

\begin{gather}
    \begin{bmatrix}
        \mathbf{Y} \\
        \beta
    \end{bmatrix}
    = \begin{bmatrix} X \\ I_p \end{bmatrix} \beta + \begin{bmatrix} I_n \\ 0 \end{bmatrix}  \mathbf{E}
    \sim \N \left(
        \begin{bmatrix}
        0 \\
        \vdots \\
        \vdots \\
        0 \\
        0 \\
        \vdots \\
        0
        \end{bmatrix},
        \left[
        \begin{array}{ c:c c c }
            \begin{matrix}
                & & \\
                & & \\
                & & \\
                & X \Sigma_p X^{\top} + \Sigma_e & \\
                & & \\
                & & \\
                & & \\
            \end{matrix}
            & \begin{matrix} \\ \\ \\ X \Sigma_p  \\ \\ \\ \end{matrix} \\
            \hdashline \\
            \begin{matrix} &  \Sigma_p X^{\top} & \end{matrix} & \Sigma_p
        \end{array}
        \right]
        \right)
    = p(\mathbf{Y}, \beta | X)
\end{gather}

with $\Sigma_p X^{\top} + \Sigma_e \in \mathbb{R}^{n\times n}$ and $\Sigma_p X^{\top} \in \mathbb{R}^{p\times n}$.

To find now the posterior distribution $p(\beta | \mathbf{Y}, X)$ one can use the rules for deriving conditional
distributions for multivariate Gaussian's presented in theorem \ref{thrm:Gaussian-Conditioning}.

\begin{theorem}\label{thrm:Gaussian-Conditioning} (\citeauthor{von_mises_mathematical_1964})

Let $A \sim \N(\mu_A, \Sigma_{AA})$ and $B \sim \N(\mu_B, \Sigma_{BB})$ be
Gaussian random vectors with the following joint distribution:

\begin{gather*}
    p(A, B) = \N \left(
    \begin{bmatrix}
        \mu_A \\
        \mu_B
    \end{bmatrix},
    \begin{bmatrix}
        \Sigma_{AA} & \Sigma_{AB} \\
        \Sigma_{BA} & \Sigma_{BB}
    \end{bmatrix}
    \right)
\end{gather*}

Then the conditional distribution $p(\mathbf{B} | \mathbf{A}=a)$ is also normally distributed
with mean $\bar{\mu}$ and covariance $\bar{\Sigma}$ of the following form:

\begin{align*}
    \bar{\Sigma} = \Sigma_{B B} - \Sigma_{B A} \Sigma_{A A}^{-1} \Sigma_{A B} & & \bar{\mu} = \mu_{B} + \Sigma_{BA} \Sigma_{AA}^{-1}(a - \mu_A)
\end{align*}


\end{theorem}



Using theorem \ref{thrm:Gaussian-Conditioning} the posterior distribution over $\beta$ is then given by:
\begin{gather*}
    p(\beta | \mathbf{Y}=y, X) \sim \N(\bar{\mu}, \bar{\Sigma}), \\
    \bar{\Sigma} = \Sigma_{p} - \Sigma_p X^{\top}(X \Sigma_p X^{\top} + \Sigma_e)^{-1} X  \Sigma_p, \\
    \bar{\mu} = \mu_{\beta} + \Sigma_p X^{\top}(X \Sigma_p X^{\top} + \Sigma_e)^{-1}y
\end{gather*}

The expression for the posterior mean and covariance matrix can be further simplified using Woodbury matrix identity
and we obtain:
\begin{align}\label{def:conditional-mean-var}
    \bar{\Sigma} = (X^{\top}\Sigma_e^{-1}X + \Sigma_p^{-1})^{-1} & & \bar{\mu} = \bar{\Sigma} X^{\top} \Sigma_e^{-1} y
\end{align}

Since $f(x) = x^{\top}\beta$, one can use the posterior mean and covariance matrix from
\ref{def:conditional-mean-var} to obtain the predictive distribution of $f^{\ast} := f(x^{\ast})$ at $x^{\ast}$
given our observations:
\begin{align}\label{def:predictive-dist}
    p(f^{\ast} | \mathbf{Y}, X, x^{\ast}) = \N(x^{\ast^{\top}} \bar{\mu}, x^{\ast^{\top}} \bar{\Sigma} x^{\ast})
\end{align}

One can also use the rules for conditioning to directly derive $f^{\ast} | \mathbf{Y}, X, x^{\ast}$.
Similar to before we can write the joint distribution $p(\mathbf{Y}, f^{\ast}| X, x^{\ast})$:

\begin{gather}
    \begin{bmatrix}
        \mathbf{Y} \\
        f^{\ast}
    \end{bmatrix}
    = \begin{bmatrix} X \\ x^{\ast} \end{bmatrix} \beta + \begin{bmatrix} I_n \\ 0 \end{bmatrix}  \mathbf{E}
    \sim \N \left(
        \begin{bmatrix}
        0 \\
        \vdots \\
        \vdots \\
        0 \\
        0
        \end{bmatrix},
        \left[
        \begin{array}{ c:c c c }
            \begin{matrix}
                & & \\
                & & \\
                & & \\
                & X \Sigma_p X^{\top} + \Sigma_e & \\
                & & \\
                & & \\
                & & \\
            \end{matrix}
            & \begin{matrix} \\ \\ \\ X \Sigma_p x^{\ast} \\ \\ \\ \end{matrix} \\
            \hdashline \\
            \begin{matrix} &  x^{\ast^{\top}}  \Sigma_p X^{\top} & \end{matrix} & \Sigma_p
        \end{array}
        \right]
        \right)
    = p(\mathbf{Y}, f^{\ast}| X, x^{\ast})
\end{gather}

The expression in \ref{def:predictive-dist} can then be derived using theorem \ref{thrm:Gaussian-Conditioning} on
conditioning of multivariate Gaussian's.

The next section will extend the Bayesian approach to non-parametric models and illustrate how Bayesian linear
regression is just a special case of GP regression.

\section{Bayesian Linear Regression as Gaussian Process Regression}\label{sec:gaussian-process-regression}
The linear model discussed so far, with a cyclic component represented by a cosine and a linear trend component,
might be an evident first guess.
However, it is unlikely that the BP values are exactly following this pattern.
Instead of reducing the function space to this specific class of linear functions, we may use our domain knowledge
to tell which functions of the infinite space of all functions are more likely to have generated our data.
As these functions are not characterized with explicit sets of
parameters, this approach belongs to the branch of non-parametric modelling.
By abandoning the parameters $\beta$, Gaussian process regression
directly aims for the predictive distribution of $f^{\ast} := f(x^{\ast})$ at an input $x^{\ast}$ given our observations.

Starting with the Bayesian linear regression example from last section and transforming it into a GP regression
problem, we recall that the distribution of $F_X = [f(x_1) \dots f(x_n))]^{\top}$ with given $X = [x_1 \dots x_n]^{\top}$ is:
\begin{gather*}
    F_X \sim \N(0,  X \Sigma_p X^{\top})
\end{gather*}

Alternatively this can be written as a distribution over the function $f(x)$:

\begin{gather*}
    f(x) \sim GP(0, k(x, x'))
\end{gather*}
where $k(x,x')$ needs to be chosen such that for an input X we obtain $K_{XX} =  X \Sigma_p X^{\top}$.
Given $\Sigma_p = \sigma_p I$, we would choose $k(x,x') = \sigma_p x^{\top} x'$, with the
input pairs $x$ and $x'$ only entering as a dot product.


%Note that since $\Sigma_p$ is postive definite we can define $\Sigma_p^{\frac{1}{2}} = (\Sigma_p^{\frac{1}{2}})^2=\Sigma_p$.
%Then defining $\phi(x) = \Sigma_p^{\frac{1}{2}} x$ the kernel function becomes $k(x, x') = \phi(x)^{\top} \phi(x')$,
%which is again the dot product of pairs of $\phi(x)$.
%This shows that Baysian linear regression with transformed inputs $\phi(x)$ and prior covariance matrix $\Sigma_p = I$,
%has the same effect as choosing a more complicated $\Sigma_p$ and leaving the inputs untouched.


%For example if we assume $\Sigma_p = \sigma_p I$, we would choose $k(x,x') = \sigma_p x^{\top} x'$.

%Assuming $\Sigma_p = \sigma_p I$ and $\Sigma_e = \Sigma_e I)$ we get for the kernel function:
%
%\begin{gather*}
%    k(x, x') = \sigma_p x^{\top} x' +  \mathbbm{1}_{x = x'}\Sigma_e
%\end{gather*}

%δ pq is a Kronecker delta which is one iff p = q and zero otherwise.

Combining $f^{\ast}$ and $\mathbf{Y}$ into a single random vector we can use the theorem \ref{thrm:Gaussian-Conditioning}
to arrive at the same posterior predictive distribution
$p(f^{\ast} | \mathbf{Y}, X, x^{\ast})$ as presented in \ref{def:predictive-dist}.
The joint distribution of $f^{\ast}$ and $\mathbf{Y}$ can be expressed as follows:

\begin{gather}
    \begin{bmatrix}
        \mathbf{Y} \\
        f^{\ast}
    \end{bmatrix}
    \sim \N \left(
        \begin{bmatrix}
        0 \\
        0
        \end{bmatrix},
        \begin{bmatrix}
        K_{XX} + \Sigma_e & K_{Xx^{\ast}} \\
        K_{x^{\ast}X} & K_{x^{\ast}x^{\ast}}
        \end{bmatrix}
        \right)
    = p(\mathbf{Y}, f^{\ast}| X, x^{\ast})
\end{gather}

where:
\begin{gather*}
    K_{XX} =
    \begin{bmatrix}
        k(x_1, x_1) & k(x_1, x_2) & \dots & k(x_1, x_n)\\
        \vdots  &  & \vdots  & \vdots \\
        k(x_n, x_1)  & k(x_n, x_1) & \dots  & k(x_n, x_n)
    \end{bmatrix}, \\
    K_{Xx^{\ast}} = K_{x^{\ast}X}^{\top} =
    \begin{bmatrix}
        k(x_1, x^{\ast}) \\
        \vdots \\
        k(x_n,  x^{\ast})
    \end{bmatrix} \text{ and }
    K_{x^{\ast}x^{\ast}} = k(x^{\ast}x^{\ast})
\end{gather*}

\subsection{Time Series Gaussian Process Regression}

Unlike in chapter \ref{ch:time-series-decomposition-and-regression}, $f(x)$ is no longer assumed to be a
deterministic and parametric function.
This way, GP regression allows us to treat $\mathbf{E}$ not simply as an error
term but an actual part of our signal which we can predict. If $\mathbf{E}$ is not independent noise but for example a
time series, where the elements of $\mathbf{E}$ are correlated, we want to leverage the information we have about an
unobserved time point given our observations.
Hence, we are not interested in the posterior distribution of $f^{\ast}$ only, but also of
$Y^{\ast} := Y(x^{\ast}) = f(x^{\ast}) + E(x^{\ast})$.

Recall the expression for the marginal likelihood $p(\mathbf{Y}| X)$ from \ref{eq:marginal-likelihood}:
\begin{gather*}
    \mathbf{Y}|X \sim \N(0,  X \Sigma_p X^{\top} + \Sigma_e) \\
\end{gather*}

Alternatively, this can be expressed as a distribution over the function $Y(x)$:
\begin{gather*}
    Y(x) \sim GP(0, k(x, x'))
\end{gather*}
The kernel function $k(x,x')$ needs to be chosen such that for an index set X we obtain $K_{XX} =  X \Sigma_p X^{\top} + \Sigma_e$.
One can then follow again the same procedure as before and combine $Y^{\ast}$ and $\mathbf{Y}$ into a single random vector:

\begin{gather}
    \begin{bmatrix}
        \mathbf{Y} \\
        Y^{\ast}
    \end{bmatrix}
    \sim \N \left(
        \begin{bmatrix}
        0 \\
        0
        \end{bmatrix},
        \begin{bmatrix}
        K_{XX} & K_{Xx^{\ast}} \\
        K_{x^{\ast}X} & K_{x^{\ast}x^{\ast}}
        \end{bmatrix}
        \right)
    = p(\mathbf{Y}, Y^{\ast}| X, x^{\ast})
\end{gather}

The predictive distribution $p(Y^{\ast} | \mathbf{Y}, X, x^{\ast})$ is then again derived by conditioning.

One could also assume additional measurement noise on the time series $f(x) + E(x)$.
We then have for the observed time series $Y(x)$:
\begin{align*}
    Y(x) = f(x) + E(x) + \epsilon   && \epsilon \sim \N(0, \sigma_n^{2})
\end{align*}
To be inline with the literature on Gaussian process regression, we will from now on consider
our goal to find some function $f(x)$, which is a combination of the mean function, until now denoted by $f(x)$,
and the stationary time series $E(x)$.
The observed time series $Y(x)$ will thus be equivalent to $f(x)$ up to some additive independent noise term $\epsilon$.
We can write:
\begin{align*}
    Y(x) = f(x) + \epsilon && \epsilon \sim \N(0, \sigma_n^{2})
\end{align*}

Assuming the same linear model as before, we have for $F_X = [f(x_1), \dots f(x_n)]^{\top}$:
\begin{gather*}
    F_X = X \beta + \mathbf{E}, \text{ with $\beta \sim \N(0, \Sigma_p)$ and $\mathbf{E} \sim \N(0, \Sigma_e)$}
\end{gather*}
%with $\mathbf{E} = [E(x_1) \dots E(x_n)]^{\top} \sim \N(0, \Sigma_e)$ for some input $x_1 \dots x_n$.
%
%For some inputs $(x_i: i = 1 \dots n)$ we assume:
%\begin{gather*}
%    f(x_i) = x_i^{\top}\beta + E(x_i) \\
%    \text{with $\beta \sim \N(0, \Sigma_p)$} \\
%    \text{and  $\mathbf{E} = [E(x_1) \dots E(x_n)]^{\top} \sim \N(0, \Sigma_e$),}
%\end{gather*}

%For some inputs $(x_i: i = 1 \dots n)$ we assume:
%\begin{gather*}
%    f(x_i) = x_i^{\top}\beta + E(x_i) \\
%    \text{with $\beta \sim \N(0, \Sigma_p)$ and  $\mathbf{E} = [E(x_1) \dots E(x_n)]^{\top} \sim \N(0, \Sigma_e$),}
%\end{gather*}
Analogously we can write:
\begin{gather*}
    f(x) \sim GP(0, k(x, x')),
\end{gather*}
with $k(x,x')$ such that for an input $X = [x_1 \dots x_n]^{\top}$ we obtain $K_{XX} =  X \Sigma_p X^{\top} + \Sigma_e$.

The joint distribution of $\mathbf{Y}$ and $f^{\ast} := f(x^{\ast})$ is given by:
\begin{gather}
    \begin{bmatrix}
        \mathbf{Y} \\
        f^{\ast}
    \end{bmatrix}
    \sim \N \left(
        \begin{bmatrix}
        0 \\
        0
        \end{bmatrix},
        \begin{bmatrix}
        K_{XX} + \sigma_n^2 I & K_{Xx^{\ast}} \\
        K_{x^{\ast}X} & K_{x^{\ast}x^{\ast}}
        \end{bmatrix}
        \right)
    = p(\mathbf{Y}, f^{\ast}| X, x^{\ast})
\end{gather}


The posterior (or predictive) distribution over $f^{\ast}$ can then again be derived by conditioning:
\begin{gather}\label{eq:posterior-gp}
    p(f^{\ast}| \mathbf{Y}, X) = \N(
K_{x^{\ast}X} (K_{XX} + \sigma_n^2 I)^{-1} \mathbf{Y},
K_{x^{\ast}x^{\ast}} - K_{x^{\ast}X}(K_{XX} + \sigma_n^2 I)^{-1}K_{Xx^{\ast}})
\end{gather}

If we are interested in predicting $Y(X)$, i.e. the iid gaussian noise term $\epsilon$
should be included in the prediction. We choose $k(x,x')$ such that
$K_{XX} =  X \Sigma_p X^{\top} + \Sigma_e + \sigma_n^2 I$.
The predictive distribution over
$Y^{\ast} := Y(x^{\ast})$ is then simply:
\begin{gather}\label{eq:posterior-gp-noise}
    p(Y^{\ast}| \mathbf{Y}, X) = \N(
K_{x^{\ast}X} K_{XX}^{-1} \mathbf{Y},
K_{x^{\ast}x^{\ast}} - K_{x^{\ast}X} K_{XX}^{-1} K_{Xx^{\ast}})
\end{gather}


%Similarly if we are interested in predicting $Y(X)$, i.e. the iid gaussian noise term $\epsilon$
%should be included in the prediction, we can write:
%
%\begin{gather*}
%    Y(x) \sim GP(0, k(x, x') + \delta_{x,x'}\sigma_n)),
%\end{gather*}
%where $k(x,x')$ is the same as above with $K_{XX} =  X \Sigma_p X^{\top} + \Sigma_e$ and $\delta_{x,x'}$ is the Kronecker delta which
%is one if $x = x'$ and zero otherwise.
%We define a new kernel function $\tilde{k}(x,x') := k(x, x') + \delta_{x,x'}\sigma_n$, then
%$\tilde{K}_{XX} =  X \Sigma_p X^{\top} + \Sigma_e + \sigma_n^2 I$ and the predictive distribution over
%$Y^{\ast} := Y(x^{\ast})$ is then:
%\begin{gather}\label{eq:posterior-gp-noise}
%    p(Y^{\ast}| \mathbf{Y}, X) = \N(
%\tilde{K}_{x^{\ast}X} \tilde{K}_{XX}^{-1} \mathbf{Y},
%\tilde{K}_{x^{\ast}x^{\ast}} - \tilde{K}_{x^{\ast}X} \tilde{K}_{XX}^{-1} \tilde{K}_{Xx^{\ast}})
%\end{gather}



Also note how until now we have still assumed $\Sigma_e$, the covariance matrix of $\mathbf{E}$, to be known.
However, deriving $\Sigma_e$ for an ARMA process with irregularly spaced samples is not straight forward, as has already
been shown in chapter \ref{ch:time-series-decomposition-and-regression}.
The next section will illustrate how choosing a specific kernel function solves this problem.

\section{Mean Function}\label{subsec:mean-function}


A Gaussian process is fully specified by its mean function, $\mu(x)$, and its covariance
function, $k(x, x')$.
However, the mean function can always be subtracted from the observed data without changing
the covariance structure of the data. In other words, if we subtract the mean function from the
observed data, we obtain a new dataset with zero mean, but the same covariance structure.


Incorporating Explicit Basis Functions \citeauthor{rasmussen_gaussian_2006} p.27


Assuming we want to model $Y(x) = f(x) + \epsilon$, with $f(x) = m(x) + E(x)$ where $m(x)$ is a deterministic mean function
and the $E(x)$ is a time series process and $\epsilon$ is some iid. gaussian noise term with variance $\sigma_n^{2}$.
%In matrix notation for an input $X$:
%
%$$ \mathbf{Y} = \matthbf{M} + \mathbf{E} + \mathbf{\epsilon}$$

Then we can model $f(x)$ with a GP:
$$ f(x) \sim GP(m(x), k(x,x'))$$

By using the conditioning rule we arrive at the following predictive distribution for $f^{\ast} := f(x^{\ast})$:
\begin{gather*}
    p(f^{\ast}| \mathbf{Y}= y, X, x^{\ast}) = N(\bar{\mu}, \bar{\Sigma}), \\
    \bar{\mu} = m(x^{\ast}) + K_{x^{\ast}X} (K_{XX} + \sigma^{n} I )^{-1}(y - m(X)),\\
    \bar{\Sigma} = K_{x^{\ast}x^{\ast}} - K_{x^{\ast}X} (K_{XX} + \sigma^{n} I )^{-1} K(X, x^{\ast})
\end{gather*}
Note how the predictive variance $\bar{\Sigma}$ is not affected by $m(x)$.
If instead a GP is fitted to $f(x) - m(x) = E(x)$ we can write:
$$ E(x) = f(x) - m(x) \sim GP(0, k(x,x'))$$
The predictive distribution over $E^{\ast} := E(x^{\ast})$ given $\mathbf{Z} := \mathbf{Y} - m(X)$ is then:
\begin{gather*}
    p(E^{\ast}| \mathbf{Z} = z, X, x^{\ast}) = N(\bar{\mu}_{E^{\ast}}, \bar{\Sigma}), \\
    \bar{\mu}_{E^{\ast}} = K_{x^{\ast}X} (K_{XX} + \sigma^{n} I )^{-1} z,\\
\end{gather*}
Since $ z = y - m(X)$, the predictive distribution over $f^{\ast}$ is recovered by adding $m(x^{\ast})$ to
the predictive mean $\bar{\mu}_{E^{\ast}}$. The predictive variance $\bar{\Sigma}$ remains unchanged,
since it is not affected by the observations nor by $m(x)$.

%In order to find $m(x)$ we again could assume the liner model from chapter \ref{ch:time-series-decomposition-and-regression}
%and use the GP only to model the errors $E(x)$.
%When fitting the model, one could optimize over the parameters $\beta$ jointly with
%the hyperparameters of the covariance function.
%However, we can also put a prior over the parameters $\beta$
%and fit a GP to $f(x)$, as has been done in section \ref{sec:gaussian-process-regression}.
Hence, when
having some knowledge about $m(x)$ one should subtract it first before fitting a GP
with a zero mean prior.

If unknown one can optimize over the paramters of the mean function and the hyperparameters
of the convariance function jointly.

Also in the case of $m(x)$ being a constant $c$ one one can simply add $c^2$ to the covariance function
and have $m(x) =0$

$f(x) \sim GP(m(x)=0, k(x_i, x_j))$

that is to say,

$E[f(x)] = 0$, $cov[f(x_i), f(x_j)] = k(x_i, x_j)$

If a constant c is added to the kernel,

    $cov[f(x_i), f(x_j)] = E[(f(x_i) - E[f(x_i)])(f(x_j) -E[f(x_i)])]= E[f(x_i)f(x_j)] - m(x_i)m(x_j) = E[f(x_i)f(x_j)] = k(x_i, x_j) + c$

So actually it is the same as a GP with

    $sqrt(c), k(x_i, x_j)$

because

    $cov[f(x_i), f(x_j)] = E[f(x_i)f(x_j)] - m(x_i)m(x_j) = E[f(x_i)f(x_j)] - c = k(x_i, x_j)$





\section{Kernel Functions}\label{subsec:kernel}

In the last section we started of with a describing the prior distribution over
$\mathbf{Y}$ or $F_X = [f(x_1) \dots f(x_n))]^{\top}$ and shoved that a kernel function $k(x, x')$ needs to be
chosen to match this distribution.
However, in Gaussian process regression it generally goes the other way around.
One would choose a prior distribution over $f(x)$ or $Y(x)$ first, which boils down to choosing a kernel function.
The kernel function evaluated at your inputs $X=[x_1 \dots x_n]^{\top}$ is then needed to calculate the
predictive distribution of $f^{\ast}$ or $y^{\ast}$.

The choice of kernel function depends on the assumptions about correlation in your output given arbitrary input pairs
$x$ and $x'$.


\subsection{Stationary Covariance Function}
Does only depend on $\tau = x - x'$.

\subsubsection{The Matérn Class of Covariance Functions}

A expression for the Matérn covariance function is given by \citeauthor{rasmussen_gaussian_2006}:

\begin{gather*}
    k_{\nu}(\tau) = \sigma^2 \frac{2^{1-\nu}}{\Gamma(\nu)}(\frac{\sqrt{2\nu} \tau}{l})^{\nu} K_{\nu}
    (\frac{\sqrt{2\nu} \tau}{l})
\end{gather*}
where $\nu$ and $l$ are positive parameters, $K_{\nu}$ is a modified Bessel function and
$\sigma^2$ = $k_{\nu}(0)$

For $\nu = r + 1/2, r \in \mathbb{N}$ the expression for the Matérn covariance function can be simplified:

\begin{gather}\label{kernel-matern}
    k_{\nu=r+1/2}(\tau) = \sigma^2 \exp(-\frac{\sqrt{2r + 1} \tau}{l}) \frac{r!}{(2p)!}
    \sum_{i=0}^{r} \frac{(r+i)!}{i!(r-i)!}(\frac{2 \sqrt{2 r + 1} \tau}{l})^{r-i}
\end{gather}


Or also with \texttt{includegraphics}:
\begin{figure}[hbt!]%--- Picture 'H'ere, 'B'ottom or 'T'op; '!' Try to
                    %impose your will to LaTeX
  \centering
  \includegraphics[width=\textwidth]{Matern_3} %<< no file extension
  %%         --- .5\textwidth stands for 50% of text width
  \caption[Geyser data: binned histogram, Silverman's and another
  kernel]%<<-- Legend for the list of figures at the beginning of you thesis
  {Matérn kernel function and sample path for different $\nu$}% legend displayed below the graph.
  \label{fig:matern}
\end{figure}


Setting $\nu = 1/2$ with input domain $X \subset \mathbb{R}$ gives raise to a continuous-time AR(1) process,
also called Ornstein-Uhlenbeck process.
With $\nu = 1/2$, i.e. $r=0$, the Matérn covariance function is given by:
\begin{gather}\label{kernel-matern-ar1}
    k(\tau) = \sigma^2 exp(- \tau/l)
\end{gather}

The autocovariance function of an Ornstein-Uhlenbeck process can be derived by solving the stochastic differential equation (SDE) that defines the process.

Starting with the SDE for an OU process:

$$dX_t = \theta (\mu - X_t)dt + \sigma_w dW_t,$$

where $X_t$ is the value of the process at time $t$, $\theta$ is a positive constant that determines the speed of mean reversion,
$\mu$ is the long-term mean of the process, $\sigma_w$ is the standard deviation of the random shocks, and $W_t$ is a standard Wiener process or Brownian motion.

The solution to the SDE is:

$$ X_t = X_0 e^{-\theta t} + \mu (1-e^{-\theta t}) +
\sigma_w e^{-\theta t} \int_{0}^{t} e^{\theta s} dW_s$$

%
%Let $Y_t = X_t - \mu$, then we have:
%
%$$dY_t = dX_t$$
%$$= \theta (\mu - X_t)dt + \sigma_w dW_t$$
%$$= \theta (-Y_t)dt + \sigma_w dW_t$$
%
%This is a linear stochastic differential equation with constant coefficients, which can be solved using the method of integrating factors. We multiply both sides of the equation by $e^{\theta t}$ and integrate from 0 to $t$:
%
%$$\int_{0}^{t} e^{\theta s} dY_s = -\int_{0}^{t} \theta e^{\theta s} Y_s ds + \int_{0}^{t} \sigma_w e^{\theta s} dW_s$$
%
%Using the fact that $\int_{0}^{t} \sigma_w e^{\theta s} dW_s$ is a Gaussian random variable with mean 0 and variance $\frac{\sigma_w^2}{2\theta}(1 - e^{-2\theta t})$, we can solve for $Y_t$ and get:
%
%$$Y_t = e^{-\theta t} Y_0 + \frac{\sigma_w}{\sqrt{2\theta}}\int_{0}^{t} e^{-\theta(t-s)} dW_s$$
%
%where $Y_0 = X_0 - \mu$.
%
%Now, we can use the definition of covariance to find the autocovariance function of the OU process:
%
%$$Cov(X_t, X_{t-k}) = E[(X_t - \mu)(X_{t-k} - \mu)]$$
%
%$$= E[(Y_t + \mu)(Y_{t-k} + \mu)]$$
%
%$$= E[Y_t Y_{t-k}] + \mu^2$$
%
%Substituting the expression for $Y_t$ and $Y_{t-k}$ from the above equation, we get:
%
%$$Cov(X_t, X_{t-k}) = e^{-\theta k} Var(Y_0) = \frac{\sigma_w^2}{2\theta} e^{-\theta k}$$
%
%where we have used the fact that $Var(Y_0) = \frac{\sigma_w^2}{2\theta}$.
The process is stationary if $\theta > 0$.
The autocovariance function of an OU process is given by
$Cov(X_t, X_{t-k}) = \frac{\sigma_w^2}{2\theta} e^{-\theta k}$,
where $k\geq 0$ and $\theta > 0$.

This is the same expression as we have obtained in \ref{kernel-matern-ar1}, where
$k(0) = \sigma^2 = \frac{\sigma_w^2}{2\theta}$ and $l=1/\theta$

To see how the Ornstein-Uhlenbeck can be considered a continuous time analogue to the discrete time
AR(1) process one can use the Euler-Maryuama discretization of the process.
Considering again the SDE for an OU process:
$$dX_t = \theta (\mu - X_t)dt + \sigma_w dW_t,$$
The process can be discretized at times $(k \Delta t)_{k \in \mathbb{N}_0}$:

$$ X_{k+1} - X_k = \theta \mu \delta t - \theta X_k \Delta t + \sigma_w (W_{k+1} - W_k)$$

The random variables $(W_{k+1} - W_k)$ are independent and identically distributed normal random variables
with expected value zero and variance $\Delta t$.
Therefore, we can set $\sigma_w (W_{k+1} - W_k) = \sigma_w \sqrt{\Delta t} \epsilon$ with $\epsilon \sim \N(0,1)$
to obtain the following recursion:
$$ X_{k+1} = \theta \mu \Delta t - (\theta \Delta t - 1) X_k + \sigma_w \sqrt{\Delta t} \epsilon$$

The recursion for an AR(1) process is:
$$ X_{k+1} = c + a X_k + b \epsilon$$
Which is identical to the expression above if $c= \theta \mu \Delta t$, $a=1- \theta \Delta t$ and
$b= \sigma_w \sqrt{\Delta t}$



More generally a continuous AR(p) process has Matérn covariance function with $\nu = p - 1/2$.
TODO: different kernels, stationary kernels, link to power spectral density, with plots
A more detailed description of covariance functions and their property can be found
in \citeauthor{rasmussen_gaussian_2006} (chapter 4).

\section{Performance Assessment}\label{sec:performance-assessment}
Inference, in the case of Gaussian process regression, revolves around the posterior (predictive) distribution
of the response variable.
To evaluate how effectively the predictive distribution explains the observed values
$\mathbf{y^{\ast}}$,
it is common practice to calculate the probability of these values based on the predictive distribution.
Equation \ref{eq:posterior-gp-noise} presents an expression for the predictive distribution of
$\mathbf{Y^{\ast}}:= [Y(x_1^{\ast}), \dots, Y(x_k^{\ast})]^{\top}$ at arbitrary inputs
$X^{\ast} = [x_1^{\ast}, \dots, x_k^{\ast}]$.
Expanding the expression from \ref{eq:posterior-gp-noise} we obtain:

\begin{gather}\label{eq:predictive-dist}
    \log p(\mathbf{Y^{\ast}} = \mathbf{y^{\ast}}| \mathbf{Y}, X) =
    -\frac{k}{2} \log 2 \pi - \frac{1}{2} \log|\bar{\Sigma}| -
        \frac{1}{2}(\mathbf{y^{\ast}} - \bar{\mu})^{\top} \bar{\Sigma}^{-1} (\mathbf{y^{\ast}} - \bar{\mu})
\end{gather}
where
$\bar{\Sigma} = K_{X^{\ast}X^{\ast}} - K_{X^{\ast}X} K_{XX}^{-1} K_{XX^{\ast}}$
and $\bar{\mu} = K_{X^{\ast}X} K_{XX}^{-1} \mathbf{Y}$.

The higher the log probability, the better the fit to the data.
In contract to other performance metrics it accounts for the complete predictive
distribution rather than just a point estimate.
For instance, when employing the sum of squared errors between the true values $y^{\ast}$ and the predictive mean
$\bar{\mu}$, the predictive covariance matrix $\bar{\Sigma}$ is completely ignored.

\section{Model Selection}

Model selection in Gaussian process regression involves identifying the optimal covariance function
along with the optimal hyperparameters.
Two common approaches for model selection are cross-validation, using a performance-based loss
function as discussed in Section \ref{sec:performance-assessment}, and Bayesian model selection,
which will be explored in the subsequent subsections.
The concepts and ideas discussed in this section are primarily derived from Chapter 5 of
the textbook from \citeauthor{rasmussen_gaussian_2006}.

\subsection{Bayesian Model Selection}

Bayesian model selection aims to find the most probable model given the available data
using a hierarchical specification of the model.
In a parametric model setting, the lowest level consists of the parameters $\beta$,
followed by the hyperparameters $\theta$, which control the parameter distribution.
The highest level encompasses the set of possible model structures $M_i$.

The posterior distribution over the parameters $\beta$ is determined using Bayes' rule:
\begin{gather*}
    p(\beta | \mathbf{Y}, X, \theta, M_i) = \frac{p( \mathbf{Y}| X, \beta,
        M_i)p(\beta|\theta, M_i)}{p(\mathbf{Y}|X, \theta, M_i)}
\end{gather*}

Here, $p(\mathbf{Y} | X, \beta, M_i)$ represents the likelihood, $p(\beta | \theta, M_i)$ denotes the prior,
and $p(\mathbf{Y} | X, \theta, M_i)$ represents the marginal likelihood.


However, in the non-parametric setting of Gaussian processes, the parameter $\beta$ does not exist and is
replaced by the function $f$ itself.
Consequently, at the lowest level, the distribution over the function $f$ is modeled using a Gaussian process,
parametrized by $\theta$.
Similarly to the parametric setting, the posterior distribution over the function values
$f^{\ast} = f(x^{\ast})$ at some arbitrary input $x^{\ast}$ is given by:
\begin{gather*}
    p(f^{\ast} | \mathbf{Y}, X, \theta, M_i) = \frac{p( \mathbf{Y}| f^{\ast}, M_i)p(f^{\ast} | \theta, M_i)}{p(\mathbf{Y}|X, \theta, M_i)}
\end{gather*}

Please refer to Equation \ref{eq:posterior-gp} for the derivation of the posterior distribution over the
function values $f^{\ast}$ when assuming $f \sim GP(0, k(x,x'))$.


By assuming a prior distribution over the hyperparameters $\theta$, a similar expression can be obtained for
the posterior distribution over the hyperparameters:
\begin{gather*}
    p(\theta | \mathbf{Y}, X, M_i) = \frac{p( \mathbf{Y}| X,M_i, \theta)
        p(\theta| M_i)}{p(\mathbf{Y}|X, M_i)}
\end{gather*}
Maximizing $p(\theta | \mathbf{Y}, X, M_i)$ yields the optimal hyperparameters.
However, when non-Gaussian priors are assumed for $\theta$, evaluating $p(\theta | \mathbf{Y}, X, M_i)$
can be challenging.
In such cases, it is common to maximize the marginal likelihood $p(\mathbf{Y} | X, \theta, M_i)$
with respect to the hyperparameters $\theta$.
This approach is equivalent to assuming uniform distributions over the hyperparameters.
The next subsection will provide more details on how to calculate and maximize the
marginal likelihood for Gaussian process regression.

Note that the scheme mentioned above can be extended to maximize the posterior over the model structures
$M_i$ in order to determine the optimal model structure.
In Gaussian process regression, this corresponds to finding the optimal kernel function type.
However, instead of directly evaluating the posterior, it is often achieved through simultaneous optimization of
the marginal likelihood with respect to the model structure $M_i$ and its hyperparameters $\theta$.
By jointly optimizing these components, we can effectively identify the most suitable kernel function for
the given problem.



%The posterior distribution over the model $M_i$ is given by:
%\begin{gather*}
%    p(M_i | \mathbf{Y}, X) = \frac{p( \mathbf{Y}| X)
%        p(M_i)}{p(\mathbf{Y}|X)}
%\end{gather*}
%where $p(\mathbf{Y}|X) = \sum_i p(\mathbf{Y}|X, M_i)p(M_i)$


\subsubsection{Marginal Likelihood}

In the context of Bayesian linear regression, the marginal likelihood expression was previously
introduced in subsection \ref{subsec:bayesian-linear-regression},
assuming a prior distribution of $p(\beta) = \mathcal{N}(0, \Sigma_p)$ and a likelihood function of
$p(\mathbf{Y} | X, \beta) = \mathcal{N}(X \beta, \Sigma_e)$.
The following expression for the marginal likelihood is obtained by marginalizing over $\beta$:

\begin{gather}\label{eq:marginal-likelihood2}
    p(\mathbf{Y}|X) = \int p(\mathbf{Y}|X,\beta) p(\beta) d\beta = \N(0, X \Sigma_p X^{\top} + \Sigma_e)
\end{gather}

Furthermore, as discussed in section \ref{sec:gaussian-process-regression},
the marginal likelihood can also be represented as a distribution over the function $Y(x)$:
\begin{gather*}
    Y(x) \sim GP(0, k(x, x'))
\end{gather*}
Here, the kernel function $k(x, x')$ is chosen such that for an index set $X$,
we obtain $K_{XX} = X \Sigma_p X^{\top} + \Sigma_e$.

By the definition of a Gaussian process, $\mathbf{Y}|X$ follows a multivariate normal distribution
with a covariance matrix of $K_{XX}(\theta)$, which is a function of the hyperparameters $\theta$.
The log marginal likelihood is hence given by:
\begin{gather}\label{eq:gaussian-marginal-likelihood}
    \log p(\mathbf{Y} | X, \theta) = - \frac{1}{2} \mathbf{Y}^{\top} K_{XX}^{-1}(\theta) \mathbf{Y} -
    \frac{1}{2} \log |K_{XX}(\theta)| - \frac{n}{2} \log 2 \pi
\end{gather}

Since the marginal likelihood already incorporates a trade-off between model fit and
model complexity, it is a suitable candidate for solving the model selection problem.
The first term, $- \frac{1}{2} \mathbf{Y}^{\top} K_{XX}^{-1}(\theta) \mathbf{Y}$,
represents a measure of the data fit. The second term, $\frac{1}{2} \log |K_{XX}(\theta)|$,
penalizes more complex models. The last term $\frac{n}{2} \log 2 \pi$ serves as a normalization constant.










%
%Although there are endless variations in the
%suggestions for model selection in the literature three general principles cover
%most: (1) compute the probability of the model given the data, using marignal likelihood.
%
%(2) estimate
%the generalization error and (3) bound the generalization error. We use the
%term generalization error to mean the average error on unseen test examples
%(from the same distribution as the training cases). Note that the training error
%is usually a poor proxy for the generalization error, since the model may fit
%the noise in the training set (over-fit), leading to low training error but poor
%generalization performance.
%
%
%
%Hyperparameters $\theta$
%
%Use marginal likelihood (or evidence):
%\begin{gather}
%    log p(\mathbf{Y} | X, \theta) = - \frac{1}{2} \mathbf{Y}^{\top} K_{XX}^{-1}(\theta) \mathbf{Y} -
%    \frac{1}{2} log \det{K_{XX}(\theta)} - \frac{n}{2} \log 2 \pi
%\end{gather}
%


%This enforces the kernel function to be symmetric and positive definite.


%which enforces
%certain properties on these kernel functions. A valid kernel function must be:
%
%\begin{itemize}
%    \item symmetric
%    \item postive definite
%\end{itemize}
%



