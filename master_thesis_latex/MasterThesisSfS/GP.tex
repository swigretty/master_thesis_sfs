\chapter{Gaussian Process Regression}\label{ch:gaussian-process-regression}

We again consider a regression problem of mapping the input $x$ to an output $f(x)$.
One approach as we have seen in seen in \ref{ch:time-series-decomposition-and-regression}
is to restrict ourselves to a
class of functions, which in our case has been the class of linear functions of the input.
Another approach is to assign a prior probability to every possible function, where higher
probabilities are assigned to functions that are assumed to be more likely.
A Gaussian process can be viewed as a gaussian distribution over functions or as an infinite set of random
variables representing the values of the function $f(x)$ at location $x$.
The Gaussian process is thus a generalization of the Gaussian distribution and a formal definition is given
by \citeauthor{rasmussen_gaussian_2006} :

\begin{definition}[Gaussian Process]\label{def:GP}
 A Gaussian process is a collection of random variables, any finite number of which have a joint Gaussian distribution.
\end{definition}


As a (multivariate) Gaussian distribution is defined by its mean and covariance matrix a gaussian process (GP) is
uniquely identified by its mean $m(x)$ and covariance function $k(x,x')$.

We write

\begin{gather}
    f(x) \sim GP(m(x), k(x,x')), \\
\end{gather}
with
\begin{gather}
    m(x) = \mathbb{E}[f(x)] \\
    k(x,x') = \mathbb{E}[(f(x)-m(x))(f(x')-m(x'))]
\end{gather}

Where where $m$ is called the mean function and k is called covariance (kernel) function, which encodes some assumption
about correlation of the response variable at different inputs.

If we assume $X$ to be the index set or set of possible inputs of $f$, then there is a random variable
$F_x := f(x)$ such that for a set $A \subset X$ with $A={x_0, \dots x_n}$ it holds that:

\[F_A = [F_{x_0}, \dots , F_{x_n}] \sim \mathcal{N}(m_A,\,K_{AA})\]
for
\begin{gather}
    K_{AA} =
    \begin{bmatrix}
        k(x_0, x_1) & k(x_1, x_2) & \dots & k(x_1, x_n)\\
        \vdots  &  & \vdots  & \vdots \\
        k(x_n, x_1)  & k(x_n, x_1) & \dots  & k(x_n, x_n)
    \end{bmatrix} \text{and }
    \mu_A =
    \begin{bmatrix}
        m(x_1) \\
        \vdots \\
        \mu(x_n)
    \end{bmatrix}
\end{gather}
The finite marginals $F_{x_0}, \dots, F_{x_n}$ of the GP thus have a multivariate gaussian distribution.
In our running example we consider $X$ to be the time interval $T_0=[0, T]$ however it could be higher dimensional.

Note that Gaussian processes with finite index sets and hence with joint gaussian distribution is just a specific case
of Gaussian process.

If we consider the linear regression case from \ref{ch:time-series-decomposition-and-regression} and assume a prior distribution
on the $\beta$'s, i.e. $\beta \sim N(0, I)$ then the predictive distribution over $\mu = X \beta$ is Gaussian:
\[
    \mu \sim N(0, XX^{\top})
\]
and represents a GP with mean function $m(x) = 0$ and kernel function $k(x, x') = x^{\top}x'$.
This special case of gaussian process regression with this specific kernel function is known as baysian linear regression.


\section{Bayesian Linear Regression}

If we consider again the linear regression model from chapter \ref{ch:time-series-decomposition-and-regression} but
we denote the mean function with $f(x)$ instead of $\mu(t)$:
\begin{align*}
    f(x) &= x^{\top}\beta, & y &= f(x) + E, & E &\sim N(0, \Sigma_n)
\end{align*}
with $x \in \mathbb{R}^p$ being again the input vector and $\beta \in \mathbb{R}^p$ is the vector with
the regression coefficients.

Assuming observations $y = [y_0 \dots y_n]^{\top}$ for inputs $X = [x_0 \dots x_n]^{\top}$
we can calculate the likelihood,
which is the probability density of the observations given $X$ and $\beta$:

\begin{gather*}
    p(y|X,\beta) =
    = \frac{1}{( (2\pi)^{n/2} \sqrt {\det(\Sigma_n))}}
    \exp(-\frac{1}{2}(y − X\beta)^{\top} \Sigma_n^{-1}(y - X\beta))
    = N(X \beta, \Sigma_n)
\end{gather*}

Unlike in chapter \ref{ch:time-series-decomposition-and-regression} we will now also
assume a prior distribution over the regression coefficients $\beta$, based on
what we believe is are likely values for the coefficients. We will choose a gaussian
distribution with mean $0$ and covariance $\Sigma_p$, i.e.:

\begin{gather*}
    p(\beta) = N(0, \Sigma_p)
\end{gather*}

Given our observations $y = [y_0 \dots y_n]^{\top}$  we can use Bayes' theorem to calculate the posterior distribution over the $\beta$s:
\begin{gather*}
    p(\beta|y, X) = \frac{p(y,\beta|X)}{p(y|X)} = \frac{p(y|X,\beta)p(\beta)}{p(y|X)}
\end{gather*}

One approach is to just plug in the expressions for $p(y|X,\beta)$ and $p(\beta|y, X)$ from above, with:

\begin{gather*}
    p(y|X) = \int p(y|X,w) p(w) dw
\end{gather*}

Or it can be helpful to combine the coefficients and the observations into a single random vector with
multivariate normal distribution:

%\begin{gather*}
%    \begin{bmatrix}
%        y_0 \\
%        \vdots \\
%        y_n \\
%        \beta_0 \\
%        \vdots \\
%        \beta_{p-1}
%    \end{bmatrix}
%    =  \begin{bmatrix}
%        X \\
%        I_p
%    \end{bmatrix}
%    \beta + E
%
%    \sim N \left(
%    \begin{bmatrix}
%        0 \\
%        \vdots \\
%        0 \\
%        0 \\
%        \vdots \\
%        0
%    \end{bmatrix},
%        \begin{bmatrix}
%        \Sigma_{yy} &  &  & \Sigma_{yw}\\
%        \\
%        \\
%        0 \\
%        \\
%    \end{bmatrix}
%
%
%    \right) = p(y, \beta)
%\end{gather*}


\[ \left|
\begin{array}{c;{2pt/2pt}r}
\mbox{\LARGE $Q$} & \begin{matrix} 0 \\ 0 \end{matrix} \\ \hdashline[2pt/2pt]
\begin{matrix} 2 & 3 \end{matrix} & -1
\end{array}
\right|
\]