\chapter{Results}\label{ch:results}

This chapter presents the performance of GP Regression in estimating various
measures based on a simulation study.
The estimation performance of GP Regression is compared with that of baseline methods.
The performance is reported for different downsampling factors, where a downsampling
factor of 2 implies that the training data contains half of the samples from the
original signal $F_X$, which has 10 samples per hour.
Additionally, results for both uniform and seasonal sampling are presented.

\section{Target Measures}

In this section, we evaluate the effectiveness of GP Regression ("gp") in
estimating specific target measures. These target measures include the one-week mean,
one-day mean, one-hour mean, and time in the target range (TTR).
We compare the estimation performance of GP Regression with that of several baseline methods,
including linear regression ("linear"), smoothing splines ("spline"), overall mean
("overall\_mean"), and naive TTR range ("naive\_ttr").

\subsection{One-Week Mean}

Figure \ref{fig:weekly-mean-performance} illustrates the performance of different
methods in estimating the one-week mean under different sampling patterns.
Subfigure \ref{fig:weekly-mean-uniform-sampling-performance} and
\ref{fig:weekly-mean-seasonal-sampling-performance} display the results
for uniform and seasonal sampling, respectively. Within each subfigure, the panels
represent decreasing downsampling factors from left to right.

Performance is presented in terms of confidence or credible interval (CI) width and CI coverage.
CI width represents the width of the estimated 95\% CI for the one-week mean BP value,
measured in mmHg. CI coverage indicates the number of times the true one-week BP
values were covered by the estimated CI over 100 simulations.

Ideally, a method should achieve 95\% CI coverage with a very low CI width,
positioning itself in the upper-left corner of the plot. Achieving 95\% coverage
is considered more critical than having a low width.
Based on this 100 simulation runs, the confidence intervals
of the CI coverage values have been calculated.
Whenever the CI of the CI coverage
spans a region above the 95\% coverage, the method is deemed adequate.

It is important to note that all plots have the same range of CI coverage values
on the y-axis, but the x-axis range of CI width values varies. As expected,
CI width values increase with larger downsampling factors and with seasonal sampling.
The CI width is capped at 30 mmHg, as larger CIs would have limited practical value.

For the largest downsampling factors, the smoothing spline interpolation method
would produce confidence intervals that exceed this limit and is therefore not
represented in the corresponding panel of subfigure
\ref{fig:weekly-mean-uniform-sampling-performance}.
It appears that the approach used for spline interpolation produces unstable
estimates during bootstrapping, resulting in very large confidence intervals.


For the one-week mean the overall mean is equivalent to the naive TTR estimate.
Therefore, naive TTR is omitted from this analysis.
All methods perform adequately with uniform sampling.
While linear regression seems to produce the smallest confidence intervals for
the largest downsampling factor, estimates do not improve as notably as for
spline interpolation and GP regression when more data becomes available.
For downsampling factors below 5, GP regression, linear regression, and spline
interpolation produce very similar results.

In the presence of seasonal sampling, the overall mean obviously does not
provide good estimates. GP regression generally
comes closest to the target coverage, but for high downsampling factors,
linear regression remains competitive and provides even smaller CI widths.
However, the more data available the more linear regression seems to
diverge from the true BP values. This effects are more pronounced in the face
of extreme seasonal sampling.


In conclusion, both linear and GP regression produce the best one-week mean estimates.
Linear regression is preferable for very high downsampling factors,
while GP regression is recommended for downsampling factors below 5,
corresponding to an average of 2 or more measurements per hour.

\begin{figure}[!htb]
\centering
\begin{subfigure}{\textwidth}
    \centering
    \includegraphics[width=\linewidth]{Pictures/final_experiment_hpdi/overall_mean_eval_sin_rbf_default}
    \caption{One-Week Mean - Uniform Downsampling}
    \label{fig:weekly-mean-uniform-sampling-performance}
\end{subfigure}

\bigskip

\begin{subfigure}{\textwidth}
    \centering
    \includegraphics[width=\linewidth]{Pictures/final_experiment_hpdi/overall_mean_eval_sin_rbf_seasonal_default}
    \caption{One-Week Mean - Seasonal Downsampling}
    \label{fig:weekly-mean-seasonal-sampling-performance}
\end{subfigure}

\bigskip

\begin{subfigure}{\textwidth}
    \centering
    \includegraphics[width=\linewidth]{Pictures/final_experiment_hpdi/overall_mean_eval_sin_rbf_seasonal_extreme}
    \caption{One-Week Mean - Extreme Seasonal Downsampling}
    \label{fig:weekly-mean-extreme-seasonal-sampling-performance}
\end{subfigure}

\caption[One-Week Mean Performance]{The Performance of different methods to
estimate the one-week mean based on different downsampling patterns. The dashed horizontal line
indicates the target coverage of 95\% 
}
\label{fig:weekly-mean-performance}
\end{figure}



\subsection{One-Day and One-Hour Mean}

Figures \ref{fig:daily-mean-performance} and \ref{fig:hourly-mean-performance}
illustrate the performance of different methods in estimating the one-day and one-hour mean,
respectively, based on various downsampling patterns.

GP regression consistently produces the best estimates for both the one-hour and
one-day mean across all sampling patterns.
Its superiority becomes more pronounced with increased data availability and
an increasing amount of seasonal sampling.

Spline interpolation, when not suffering from the estimation instability
described in the previous subsection, also yields decent results.
In the case of uniform sampling, the one-day mean is adequately approximated
by the naive TTR method.

However, linear regression does not appear to produce accurate estimates for
either the one-day or one-hour mean, regardless of the sampling pattern.


\begin{figure}[!htb]
\centering
\begin{subfigure}{\textwidth}
    \centering
    \includegraphics[width=\linewidth]{Pictures/final_experiment_hpdi/mean_24h_eval_sin_rbf_default}
    \caption{One-Day Mean Performance - Uniform Downsampling}
    \label{fig:daily-mean-uniform-sampling-performance}
\end{subfigure}

\bigskip

\begin{subfigure}{\textwidth}
    \centering
    \includegraphics[width=\linewidth]{Pictures/final_experiment_hpdi/mean_1h_eval_sin_rbf_seasonal_default}
    \caption{One-Day Mean Performance - Seasonal Downsampling}
    \label{fig:daily-mean-seasonal-sampling-performance}
\end{subfigure}

\bigskip

\begin{subfigure}{\textwidth}
    \centering
    \includegraphics[width=\linewidth]{Pictures/final_experiment_hpdi/mean_1h_eval_sin_rbf_seasonal_extreme}
    \caption{One-Day Mean Performance - Extreme Seasonal Downsampling}
    \label{fig:daily-mean-extreme-seasonal-sampling-performance}
\end{subfigure}

\caption[One-Day Mean Performance]{The Performance of different methods to
estimate the one-day mean based on different downsampling patterns.
}
\label{fig:daily-mean-performance}
\end{figure}




\begin{figure}[!htb]
\centering
\begin{subfigure}{\textwidth}
    \centering
    \includegraphics[width=\linewidth]{Pictures/final_experiment_hpdi/mean_1h_eval_sin_rbf_default}
    \caption{One-Hour Mean Performance - Uniform Downsampling}
    \label{fig:hourly-mean-uniform-sampling-performance}
\end{subfigure}

\bigskip

\begin{subfigure}{\textwidth}
    \centering
    \includegraphics[width=\linewidth]{Pictures/final_experiment_hpdi/mean_24h_eval_sin_rbf_seasonal_default}
    \caption{One-Hour Mean Performance - Seasonal Downsampling}
    \label{fig:hourly-mean-seasonal-sampling-performance}
\end{subfigure}

\bigskip

\begin{subfigure}{\textwidth}
    \centering
    \includegraphics[width=\linewidth]{Pictures/final_experiment_hpdi/mean_24h_eval_sin_rbf_seasonal_extreme}
    \caption{One-Hour Mean Performance - Extreme Seasonal Downsampling}
    \label{fig:hourly-mean-extreme-seasonal-sampling-performance}
\end{subfigure}

\caption[One-Hour Mean Performance]{The Performance of different methods to
estimate the one-hour mean based on different downsampling patterns.
}
\label{fig:hourly-mean-performance}
\end{figure}


\subsection{Time in Target Range}\label{subsec:time-in-target-range}

Figure \ref{fig:ttr-performance} displays the performance of different methods
in estimating TTR based on various downsampling patterns.
Among these methods, GP Regression stands out as the only one consistently achieving
adequate CI coverage for most of the downsampling patterns.
While smoothing spline interpolation comes close, particularly with uniform sampling and
large downsampling factors, GP regression consistently yields narrower CIs.

It is important to note that TTR does not depend on the absolute values of the blood pressure (BP)
estimates but rather focuses on whether these estimates fall within the target range.
As TTR is not affected by extreme estimate values occasionally produced by smoothing spline interpolation,
this method provides reasonable TTR estimates.

However, the current approach used for estimating TTR (naive\_ttr) performs poorly.
This is because it does not estimate TTR through the estimation of $f(x)$ but directly
operates on the noisy measurements $Y(x)$. As a result, it generally underestimates TTR.


\begin{figure}[!htb]
\centering
\begin{subfigure}{\textwidth}
    \centering
    \includegraphics[width=\linewidth]{Pictures/final_experiment_hpdi/ttr_eval_sin_rbf_default}
    \caption{TTR Performance - Uniform Downsampling}
    \label{fig:ttr-uniform-sampling-performance}
\end{subfigure}

\bigskip

\begin{subfigure}{\textwidth}
    \centering
    \includegraphics[width=\linewidth]{Pictures/final_experiment_hpdi/ttr_eval_sin_rbf_seasonal_default}
    \caption{TTR Performance - Seasonal Downsampling}
    \label{fig:ttr-seasonal-sampling-performance}
\end{subfigure}

\bigskip

\begin{subfigure}{\textwidth}
    \centering
    \includegraphics[width=\linewidth]{Pictures/final_experiment_hpdi/ttr_eval_sin_rbf_seasonal_extreme}
    \caption{TTR Performance - Extreme Seasonal Downsampling}
    \label{fig:ttr-extreme-seasonal-sampling-performance}
\end{subfigure}

\caption[TTR Performance]{The Performance of different methods to
estimate TTR based on different downsampling patterns.
}
\label{fig:ttr-performance}
\end{figure}


\section{Examples}


\subsection{Impact of Downsampling Factor}

Figure \ref{fig:ex-downsampling-factor} provides an example
of how the different methods improve with more data.
Here, linear regression
does already produce good estimates at high downsampling
factors but unlike GP regression does not improve a lot with more data.
At the same time smoothing spline fails to capture the cyclic component
due to the prominent AR component
also when a lot of data is available.

\begin{figure}
\begin{subfigure}{\textwidth}
    \centering
    \includegraphics[width=0.45\linewidth]{
       Pictures/plots_final4/sin_rbf_default_0.05/09_13_11_09_16/plot_posterior_confint_spline.pdf}
    \includegraphics[width=0.45\linewidth]{
       Pictures/plots_final4/sin_rbf_default_0.4/09_12_13_24_50/plot_posterior_confint_spline.pdf}
  \caption{Smoothing spline}
\end{subfigure}

\begin{subfigure}{\textwidth}
    \centering
    \includegraphics[width=.45\linewidth]{
  Pictures/plots_final4/sin_rbf_default_0.05/09_13_11_09_16/plot_posterior_confint_linear.pdf}
    \includegraphics[width=0.45\linewidth]{
  Pictures/plots_final4/sin_rbf_default_0.4/09_12_13_24_50/plot_posterior_confint_linear.pdf}
  \caption{Linear regression }
\end{subfigure}

\begin{subfigure}{\textwidth}
    \centering
    \includegraphics[width=0.45\linewidth]{
  Pictures/plots_final4/sin_rbf_default_0.05/09_13_11_09_16/plot_posterior_confint_gp.pdf}
    \includegraphics[width=0.45\linewidth]{
  Pictures/plots_final4/sin_rbf_default_0.4/09_12_13_24_50/plot_posterior_confint_gp.pdf}
  \caption{GP regression}
\end{subfigure}\hfill

\caption[Impact of Downsampling Factor]{Impact of Downsampling Factor:
    Prediction of true BP values $F_X$ (black)
    anc CI (gray area) form measurments (red dots) produced by
    a downsampling factor of 20 (left panels) and 2.5 (right panels).
    The true BP values are shown by the red dotted line.}
\label{fig:ex-downsampling-factor}
\end{figure}


\subsection{Seasonal Samping and Downsampling Factor}

Figure \ref{fig:ex-seasonal-sampling} show the impact of different
downsampling factors when there is extreme seasonal sampling.
While linear regression does produce already good results when there is only
very little data available it does not improve much with more data, where
it massively underestimates the uncertainty. Linear regession leads thus
to a reduced CI coverage when there is more data.
On the other hand spline regression does not provide very good
estimates of the expected BP values but since
the CI are huge, the CI coverage will not be too bad.
The predictions improve with more data, especially
at the peaks, where data density is higher than in the valeys.
Also since smoothing spline interpolation does not have any incentive
to fit a cyclic pattern the uncertainty estimates in the valey are much
higher than the one produced by GP regression and linear regression.



\begin{figure}
\begin{subfigure}{\textwidth}
    \centering
    \includegraphics[width=0.45\linewidth]{
        Pictures/plots_final4/sin_rbf_seasonal_extreme_0.05/09_13_07_35_41/plot_posterior_confint_spline.pdf}
    \includegraphics[width=0.45\linewidth]{
        Pictures/plots_final4/sin_rbf_seasonal_extreme_0.4/09_13_12_38_32/plot_posterior_confint_spline.pdf}
  \caption{Smoothing spline}
\end{subfigure}

\begin{subfigure}{\textwidth}
    \centering
    \includegraphics[width=.45\linewidth]{
        Pictures/plots_final4/sin_rbf_seasonal_extreme_0.05/09_13_07_35_41/plot_posterior_confint_linear.pdf}
    \includegraphics[width=0.45\linewidth]{
        Pictures/plots_final4/sin_rbf_seasonal_extreme_0.4/09_13_12_38_32/plot_posterior_confint_linear.pdf}
  \caption{Linear regression }
\end{subfigure}

\begin{subfigure}{\textwidth}
    \centering
    \includegraphics[width=0.45\linewidth]{
        Pictures/plots_final4/sin_rbf_seasonal_extreme_0.05/09_13_07_35_41/plot_posterior_confint_gp.pdf}
    \includegraphics[width=0.45\linewidth]{
        Pictures/plots_final4/sin_rbf_seasonal_extreme_0.4/09_13_12_38_32/plot_posterior_confint_gp.pdf}
  \caption{GP regression}
\end{subfigure}\hfill

\caption[Seasonal Samping and Downsampling Factor]{easonal Samping and Downsampling Factor:
    Prediction of true BP values $F_X$ (black)
    anc CI (gray area) form measurments (red dots) produced by extreme seasonal sampling and
    a downsampling factor of 20 (left panels) and 2.5 (right panels).
    The true BP values are shown by the red dotted line.}
\label{fig:ex-seasonal-sampling}
\end{figure}



\subsection{Dominant Cyclic Component vs. Dominant AR Component}

Figure \ref{fig:ex-ar-cyclic} compares
predictions when the AR vs. when the cyclic component is dominating.
When the cyclic component is dominating
all methods perform well.
However when the AR componen is strong, only the GP is able
to povide good local BP value estimates but also
with rather large CI.
Linear regression is bound to predict a perfect sinosoid with a linear
trend and hence has no chance to fit the AR component.
The undertainty is also very much underestimated.


\begin{figure}
\begin{subfigure}{\textwidth}
    \centering
    \includegraphics[width=0.45\linewidth]{
        Pictures/plots_final4/sin_rbf_default_0.2/09_12_13_05_43/plot_true_mean_decomposed.pdf}
    \includegraphics[width=0.45\linewidth]{
        Pictures/plots_final4/sin_rbf_default_0.2/09_12_13_08_34/plot_true_mean_decomposed.pdf}
  \caption{Decompositon of $f(x)$. Cyclic Component in orange and AR component in blue.}
\end{subfigure}

\begin{subfigure}{\textwidth}
    \centering
    \includegraphics[width=0.45\linewidth]{
        Pictures/plots_final4/sin_rbf_default_0.2/09_12_13_05_43/plot_posterior_confint_spline.pdf}
    \includegraphics[width=0.45\linewidth]{
        Pictures/plots_final4/sin_rbf_default_0.2/09_12_13_08_34/plot_posterior_confint_spline.pdf}
  \caption{Smoothing Spline}
\end{subfigure}

\begin{subfigure}{\textwidth}
    \centering
    \includegraphics[width=.45\linewidth]{
        Pictures/plots_final4/sin_rbf_default_0.2/09_12_13_05_43/plot_posterior_confint_linear.pdf}
    \includegraphics[width=0.45\linewidth]{
        Pictures/plots_final4/sin_rbf_default_0.2/09_12_13_08_34/plot_posterior_confint_linear.pdf}
  \caption{Linear regression }
\end{subfigure}

\begin{subfigure}{\textwidth}
    \centering
    \includegraphics[width=0.45\linewidth]{
        Pictures/plots_final4/sin_rbf_default_0.2/09_12_13_05_43/plot_posterior_confint_gp.pdf}
    \includegraphics[width=0.45\linewidth]{
        Pictures/plots_final4/sin_rbf_default_0.2/09_12_13_08_34/plot_posterior_confint_gp.pdf}
  \caption{GP regression}
\end{subfigure}\hfill

\caption[Dominant Cyclic Component vs. Dominant AR Component]{Dominant Cyclic Component vs. Dominant AR Component:
    Prediction of true BP values $F_X$ (black)
    and CI (gray area) form measurments (red dots). Measurments
    have been uniformly sampled with a downsampling factor of 5.
    The true BP values are shown by the red dotted line.
    The right  panels show the BP values produce by a process with prominent AR component
    while the left panel the cyclic component is prominent.
 }
\label{fig:ex-ar-cyclic}
\end{figure}

\subsection{Smoothing Spline Failure}
Figure \ref{fig:ex-spline-failure} illustrates how spline regression
produces  very
large confidence interval for high downsampling factors.
Boostrapping in these conditions leads to very instable BP estimates,
which is shown in figure \ref{subfig: ex-spline-failure-bootstrap}.
The estimated BP values $F_X$, from one boostrap sample, comprises very extreme BP values, which will
lead to underestimation of the one-week mean.




\begin{figure}[!htb]
\centering
\begin{subfigure}{.5\textwidth}
    \centering
    \includegraphics[width=\linewidth]{Pictures/spline_extreme/plot_posterior_confint_spline}
    \caption{Smoothing Spline PB estimates (black line), which is the mean over 100 botstrap samples.
    The true BP values red dotted line) are shown with training samples (red dots)}
\end{subfigure}\hfill
\begin{subfigure}{.42\textwidth}
    \centering
    \includegraphics[width=\linewidth]{Pictures/spline_extreme/plot_pred_bootstrap_spline_reg_v2_70}
  \caption[Estimate Single Bootstrap Sample]{
      BP value estimates (black line) based on samples of one bootstrap iteration (red dots). This would
  lead to an extreme underestimation of the one-week mean.}
    \label{subfig: ex-spline-failure-bootstrap}
\end{subfigure}
\caption[Smoothing Spline Failure]{
    Smoothing Spline Failure:
    Smoothing spline giving extreme local predictions of the true BP values $f(x)$
    for a downsampling factor of 2.
    The huge drop produced at the end of the one week window
    leads to huge confidence intervals}
\label{fig:ex-spline-failure}
\end{figure}






%\begin{example}
%
%\begin{figure}[!htb]
%\centering
%\includegraphics[width=\linewidth]{Pictures/plots_final2/sin_rbf_default_0.2/09_11_07_49_51/plot_posterior}
%\caption[GP Prediction]{Predictive mean of $F_{X}$ (blackline) with
%predictive variance (grey area) based on observations (red dots) }
%\label{fig:ex1-gp-prediction}
%\end{figure}
%
%\begin{figure}[!htb]
%\centering
%\includegraphics[width=\linewidth]{Pictures/plots_final2/sin_rbf_default_0.2/09_11_07_49_51/plot_mean_decomposed}
%\caption[Mean Decomposed Predicted vs True]{Each figure shows one sample $F_X$ drawn from the true GP (red dashed line) with noisy observations
%  (red dots) sampled at a frequency of 0.5/hour}
%\label{fig:ex1-gp-mean-decomposed}
%\end{figure}
%
%
%\begin{figure}[!htb]
%\centering
%\begin{subfigure}{.45\textwidth}
%    \centering
%    \includegraphics[width=\linewidth]{Pictures/plots_final2/sin_rbf_default_0.2/09_11_07_49_51/plot_posterior_spline}
%  \caption[Spline]{The sample $F_X$ shown to the right, decomposed in to the contribution of the Periodic kernel (orange),
%      Matérn kernel (blue), RBF kernel (green).}
%  \label{fig:ex1-spline}
%\end{subfigure}\hfill
%\begin{subfigure}{.45\textwidth}
%    \centering
%    \includegraphics[width=\linewidth]{Pictures/plots_final2/sin_rbf_default_0.2/09_11_07_49_51/plot_posterior_linear}
%  \caption[Linear Regression]{Each figure shows one sample $F_X$ drawn from the true GP (red dashed line) with noisy observations
%      (red dots) sampled at a frequency of 0.5/hour}
%  \label{fig:ex1-linear}
%\end{subfigure}
%\caption[Baseline Methods]{Baseline Methods Example Prediction}
%\label{fig:ex1}
%\end{figure}
%



%\end{example}









