%%%--- Template for master thesis at SfS
%%%--- Modified template with more comments and examples -- SG, 11/06/09
%%%------
\documentclass[11pt,a4paper,twoside,openright]{report}
\usepackage[english]{ETHDAsfs}%--> ETHDASA + fancyhdr + ... "umlaute"
%  + sfs-hyper -> hyperref 

\usepackage{pdfpages}%%to include the confirmation of originality (plagiarism
\usepackage{amsbsy}%% for \boldsymbol and \pmb{.}
\usepackage{amssymb}%% calls  amsfonts...
\usepackage{graphicx}%-- für PostScript-Grafiken (besser als  psfig!)
%\usepackage[draft]{graphicx} % grafics shown as boxes --> faster compilation
%
\usepackage[longnamesfirst]{natbib}%was {sfsbib}%- Für  Literatur-Referenzen
%           ^^^^^^^^^^^^^^ 1) "Hampel, Ronchetti, ..,"  2) "Hampel et al"
% Engineers (and other funny people) want to see [1], [2] 
% ---> use 'numbers' : \usepackage[longnamesfirst,number]{natbib}
%
%
\usepackage{texab}%- 'tex Abkürzungen' /u/sfs/tex/tex/latex/texab.sty
        %%- z.B.  \R, \Z, \Q, \Nat für reelle, ganze, rationale, natürl. Zahlen;
        %%-       \N   (Normalvert.)  \W == Wahrscheinlichkeit .....
        %%-  \med, \var, \Cov, \....
        %%-  \abs{x} == |x|   und   \norm{y} ==  || y ||   (aber anständig)
%% NOTE: texab contains many useful definitions and "shortcuts". It is
%% worth to open the file and have a look at them. HOWEVER, some
%% definitions are a bit can lead to conflicts with other packages. You
%% might for example want to comment out the line defininf \IF as an
%% operator when working with the algorithmic package, or to comment out
%% the line defining a command \Cite with working with the Biblatex package  
\usepackage{amsmath}
%\usepackage{mathrsfs}% Raph Smith's Formal Script font --> provides \mathscr
\usepackage[utf8]{inputenc}% <<------- Unicode, *NOT* iso-latin1 !
\usepackage{ae}% A[lmost] E[uropean] Fonts
\usepackage{enumerate}% Fuer selbstdefinierte Nummerierungen
%--------
\usepackage{relsize}%-> \smaller (etc) used here
\usepackage{color} %% to allow coloring in code listings
\usepackage{listings}% Fuer R-code, C-code, ....  and settings for these:
\definecolor{Mygrey}{gray}{0.75}% for linenumbers only!
\definecolor{Cgrey}{gray}{0.4}% for comments
\lstloadlanguages{R}
%%--- first version of "listings of R"-style : ---------------------------
% %% using \smaller here: makes R code listings use a *small* font:
% \lstset{language=R,basicstyle=\smaller[2],commentstyle=\rmfamily\smaller,
%   showstringspaces=false,xleftmargin=4ex,
%   literate={<-}{{$\leftarrow$}}1 {~}{{$\sim$}}1}
% \lstset{escapeinside={(*}{*)}} % for (*\ref{ }*) inside lstlistings (Scode) 
%\newcommand{\lil}[1]{\lstinline|#1|}
%%--- newer version of "listings of R"-style : ---------------------------
\lstset{%% Help, e.g. --> https://en.wikibooks.org/wiki/LaTeX/Source_Code_Listings
language=R,
basicstyle=\ttfamily\scriptsize,%%- \small > \footnotesize > \scriptsize > \tiny
%commentstyle=\ttfamily\color{Cgrey},
commentstyle=\itshape\color{Cgrey},
numbers=left,
numberstyle=\ttfamily\color{Mygrey}\tiny,
stepnumber=1,
numbersep=5pt,
backgroundcolor=\color{white},
showspaces=false,
showstringspaces=false,
showtabs=false,
frame=single,
tabsize=2,
captionpos=b,
breaklines=true,
%breakatwhitespace=false,
keywordstyle={},
morekeywords={},
xleftmargin=4ex, 
literate={<-}{{$\leftarrow$}}1 {~}{{$\sim$}}1}
\lstset{escapeinside={(*}{*)}} % for (*\ref{ }*) inside lstlistings (Scode) 
%%----------------------------------------------------------------------------

%%------- Theoreme ---
\newtheorem{definition}{Definition}[subsection]
\newtheorem{lemma}[definition]{Lemma}
\newtheorem{theorem}[definition]{Theorem}
\newtheorem{Coro}[definition]{Corollary}
\theoremstyle{definition} 
\newtheorem{example}[definition]{Example}
\newtheorem*{note}{Note}
\newtheorem*{remark}{Remark}

\DeclareMathOperator*{\plim}{plim}
% \def\MR#1{\href{http://www.ams.org/mathscinet-getitem?mr=#1}{MR#1}}

% \newcommand{\Lecture}[3]{\marginpar{#3.#2.#1}}
% \newcommand{\Fu}{\mathcal{F}}
\newcommand{\aatop}[2]{\genfrac{}{}{0pt}{}{#1}{#2}}

%\renewcommand{\theequation}{\arabic{equation}}
\numberwithin{equation}{subsection}

%%%%%%%%%%%%%%%%%%%%%%%%%%%%%%%%%%%%%%%%%%%%%%%%%
%%% Path for your figures                      %%%
%%%%%%%%%%%%%%%%%%%%%%%%%%%%%%%%%%%%%%%%%%%%%%%%%
% Set the paths where all figures are taken from:
\graphicspath{{Pictures/}}

%%%%%%%%%%%%%%%%%%%%%%%%%%%%%%%%%%%%%%%%%%%%%%%%%
%%% Define your own commands here             %%%
%%%%%%%%%%%%%%%%%%%%%%%%%%%%%%%%%%%%%%%%%%%%%%%%%
\newcommand{\Bruch}[2]{{}^{#1}\!\!/\!_{#2}}
\renewcommand{\labelenumi}{\roman{enumi}.)}



\begin{document}
\bibliographystyle{chicago}% ---> Hampel,F., E.Ronchetti,... W.Stahel(1986) ...
 %was \bibliographystyle{sfsbib}\citationstyle{dcu} %OR DEFAULT : \citationstyle{agsm}

\pagenumbering{roman}%- roman numbering for first few pages

%%%%%%%%%%%%%%%%%%%%%%%%%%%%%%%%%%%%%%%%%%%%%%%%%
%%% Title page                                %%%
%%%%%%%%%%%%%%%%%%%%%%%%%%%%%%%%%%%%%%%%%%%%%%%%%
\period{Summer 2023}
\dasatype{Master Thesis}
\students{Student Muster}
\mainreaderprefix{Advisor:}
\mainreader{Prof.\ Dr.\ Your supervisor}
\alternatereaderprefix{Co-Advisor}
\alternatereader{Your co-supervisor}
\submissiondate{13 March 2023}
\title{Time Series Analysis \\ for Irregularly Sampled Data }

\maketitle%- Titelseite wird abgeschlossen
\cleardoublepage
 %%~~~~~~~~~~~~~~~~~~~~~~~~~~~~~~~~~~~~~~~~

%%%%%%%%%%%%%%%%%%%%%%%%%%%%%%%%%%%%%%%%%%%%%%%%%
%%% Insert here acknowledgements and abstract %%%
%%%%%%%%%%%%%%%%%%%%%%%%%%%%%%%%%%%%%%%%%%%%%%%%%
%% Dedication (optional)
\markright{}
\vspace*{\stretch{1}}
\begin{center}
    To some special person
\end{center}
\vspace*{\stretch{2}}

% Preface (optional)
\newpage
\markboth{Preface}{Preface}
\chapter*{Preface}

% TODO uncomment
%Thank you Dr. Markus Kalisch for your exceptional
%supervision, for being very open to my ideas but also providing invaluable
%guidance, for proofreading of every page of my thesis
%and generally for your enthusiasm to discover the realm of Gaussian
%processes.
%
%Thank you, Dr. David Perruchoud for your consistent support throughout the
%thesis, for helping me to come up with the research topic and
%to always ask the questions that would keep me on track.
%
%My appreciation also goes to Dr. Josep Sola, Dr. Tiago Almeida and the
%entire Aktiia team for their valuable feedback on my research findings and for
%the office space right next to Limmat.
%
%Finally, I would like to express my gratitude to Jana Reichmann, Luca Marano,
%Moritz Ritter and Robin Siedl, for their attentive ears and moral support
%throughout the completion of this thesis.
%




% Abstract should not be longer than one page.
\newpage
\markboth{Abstract}{Abstract}
\chapter*{Abstract}

Conventional time series analysis methods typically assume observations at
evenly spaced time intervals.
In response to the challenges posed by irregularly sampled time series data and
motivated by a real-world blood pressure time series example,
this study introduces Gaussian processes as a powerful tool for analysis.

We first highlight the limitations of conventional approaches when dealing with
irregularly spaced data.
Subsequently, we demonstrate how Gaussian process regression overcomes these
limitations by enabling the modeling of time series in continuous time.

Drawing inspiration from the real-world example, we conduct a simulation study
to assess the suitability of Gaussian process regression for estimating blood
pressure values from irregularly spaced simulated measurements.
The measurements were simulated through the simulation of a blood pressure time
series that mimics real-world dynamics by
featuring cyclic, autoregressive, and long-term trend components.

We evaluate the performance of estimating clinically relevant target measures,
including time in the target range, one-week, one-day, and one-hour mean blood pressure.
Performance assessment involves calculating confidence interval coverage and width
through repeated simulations.

We compare the estimation performance of Gaussian process regression with
baseline methods, among them linear and spline regression,
considering varying data densities and sampling patterns.
These patterns include uniform and seasonal sampling, with the latter reflecting circadian rhythms.

Results reveal that Gaussian process regression consistently outperforms baseline methods
across all target measures, data densities, and sampling patterns.
Its superiority becomes more evident with increased data density and seasonal sampling,
particularly for smaller time windows like the one-hour or one-day mean blood pressure.

Linear regression exhibits its strengths in estimating the one-week mean even
under low data density conditions, but its improvement saturates with additional data.
Spline regression performs well with abundant uniformly sampled data but struggles
with seasonal sampling, where performance deteriorates as more data is added.


%\chapter*{Abstract}
%Conventional methods for time series analysis generally assume
%observations at equally spaced time points.
%Motivated by a real-world example of a blood pressure time series observed at irregularly
%spaced time points, this study aims at presenting Gaussian processes for the analysis of
%irregularly spaced time series.
%
%We firs present the limitation of a conventional method of dealing with
%irregularly spaced data. Further we show how Gaussian process regression
%can overcome these limitations, since it can model a time series in continuous time.
%
%Inspired by the real-world example a simulation study is performed that
%should show the suitability of Gaussian process regression for the purpose
%of estimating blood pressure values from irregularly spaced measurements.
%First blood pressure time series and its observations are simulated such
%that they would closely mimic the real-world blood pressure.
%Based on
%the real-world blood pressure data, the
%simulated time blood pressure time series is defined to feauture
%a cyclic component that mimics the circadian cycle, an autoregressive component
%and a long term trend component.
%Suitability is assessed by assessing the performance of estimating the expected value
%and confidence intervals of some
%clinically relevant target measures from one-week of blood pressure measurements.
%The target measures are time in target range as well as the
%one-week, one-day and one-hour mean blood pressure.
%Performance is assessed by calculating confidence interval coverage and width
%from repeated simulation of a true blood pressure
%time series.
%The estimation performance is compared to those of baseline methods, spline regressin
%and linear regression. The baseline method can work on irregularly spaced
%samples but do not  explicitly model
%the time series component. Different degree of data density and sampling patterns have
%been investigated. The uniform sampling pattern produces observations uniformly
%sampled across time.
%The seasonal sampling pattern would follow the circadian cycle, simulating
%higher data density during the day than during the night.
%
%Gaussian process regression outperforms the baseline methods considering all
%target measures, data densities and sampling patterns. The superiority
%is more pronounced when data density increases, when there is seasonal sampling
%and when considering smaller time windows such as the one-hour
%or one-day mean blood pressure. Linear regression generally produces the narrowest
%confidence interval can adequately estimate
%the one-week mean even under very low data density, while not imporoving much with
%more data. Spline regression performs well when there is plenty of data and
%unifrom sampling. Spline and linear regression both perfrom worse
%in the face of seasonal sampling and performance decrases the more data is added in this
%scenario.






%%% Local Variables: 
%%% mode: latex
%%% TeX-master: "MasterThesisSfS"
%%% End: 


%%%%%%%%%%%%%%%%%%%%%%%%%%%%%%%%%%%%%%%%%%%%%%%%%
%%% Table of contents and list of figures and %%%   
%%% tables (no need to change this usually)   %%%
%%%%%%%%%%%%%%%%%%%%%%%%%%%%%%%%%%%%%%%%%%%%%%%%%
\newpage
\tableofcontents
\newpage
\listoffigures
\newpage
\listoftables

%% Notations and glossary (optional)
\cleardoublepage
\phantomsection
\addcontentsline{toc}{chapter}{\protect\numberline{}{Notation}}
\markboth{Notation}{Notation}
\chapter*{Notation}
\label{c:Notation}

\section*{General Statements}

\begin{itemize}[label={},leftmargin=*]
  \item Prediction refers to estimation of the expected time series value at some time
  $x^{\ast}$, with $x^{\ast}$ being within the time range of available observations.
  \item Forcasting refers to estimation of the expected time series value at some time
  $x^{\ast}$, with $x^{\ast}$ being after the last available observations.
  \item Vectors are column vectors unless stated otherwise.
  \item Blood Pressure or BP always refers to the systolic blood pressure.
\end{itemize}


\section*{Abbreviation}\label{sec:abbreviation}
\begin{itemize}[label={},leftmargin=*]
  \item GP: Gaussian process.
  \item BP: (Systolic) Blood pressure.
  \item TTR: Time in target range
  \item CI: Refers to both confidence and credible interval
  \item OLS: Ordinary Least Squares.
  \item iid: Independent and identically distributed.

\end{itemize}

\section*{Symbols}\label{sec:symbols}
\begin{itemize}[label={},leftmargin=*]
  \item $\mathcal{N}(\mu,\,\sigma^{2})$ : Normal distribution with mean $\mu$ and standard deviation $\sigma$
  \item $X_1 \dots X_n \iidsim F$ : $X_1 \dots X_n$ are iid with distribution $F$
  \item $|M|$: determinant of matrix $M$
  \item $\ERW{X}$: Expectation of X
  \item $\COV{X,Y}$: Covariance between $X$ and $Y$
  \item $\VAR{X}$: Variance of X
\end{itemize}


%%% Local Variables: 
%%% mode: latex
%%% TeX-master: "MasterThesisSfS"
%%% End: 

\cleardoublepage
\pagenumbering{arabic}%--- switch back to standard numbering 


%%%%%%%%%%%%%%%%%%%%%%%%%%%%%%%%%%%%%%%%%%%%%%%%%
%%% Your text... Either write here directly,  %%%
%%% or even better: write in separate files   %%%
%%% that you just have to include here.       %%% 
%%%%%%%%%%%%%%%%%%%%%%%%%%%%%%%%%%%%%%%%%%%%%%%%%
%! Author = gianna
%! Date = 05.04.23
\chapter{Introduction}\label{ch:introduction}


\section{Thesis Objective}\label{sec:thesis-objective}

The thesis aims at presenting Gaussian process regression for modeling time
series based on irregularly spaced observations.

The topic is motivated by a "real world" problem from medicine which will
serve as a running example throughout the thesis.
%
%Motivated by a "real world" problem from medicine, I have tried to identify the most important concepts and basic
%methods for the analysis of irregularly space time series, which I will be presented in this thesis.
The problem at hand is the one of estimating certain time series properties (target measures) from a dataset
featuring systolic blood pressure (BP) measurements sampled at irregularly spaced time points.
High BP is known to be a risk factor for cardiovascular disease.
A person’s BP level is generally summarized using the average BP value over available measurements within a given time range.
A novel monitoring device already allows to collect BP estimates round the clock.
The device is collecting photoplethysmography (PPG) signals and converting them into BP measurements.
Typically, the system will yield approximately 1.5 BP measurements per hour, but depending on the quality of the PPG signal and some additional external factors,
this sampling frequency can widely vary and the expected range lies roughly between 0 and 5 measurements per hour.
Having good estimates of the true BP values at any, potentially not observed, time would allow for a better estimation
of the person’s cardiovascular risk, and enable the development of novel valuable metrics.
The thesis will focus on a set of target measures,
which have been considered most relevant for estimating the person’s cardiovascular risk:
\begin{itemize}
    \item One-hour, one-day and one-week mean BP value
    \item Time in target range
    \item Day vs night BP
\end{itemize}
Besides the point estimates also their CIs are of interest.
Importantly, the CI should be able to capture the uncertainty due to the lack of data in the proximity of the point of prediction.
This implies, that the width of the CI intervals around the mean function will not be constant over time but depend, among
other factors, on how much data is available in the proximity of a given time point.
The described endpoints are all based on prediction at the not observed passed time points however not on forcasting at new time points in
the future.
Hence, the thesis will only focus on the task of reconstructing BP values between the first and last time point in the dataset.

The standard time series analysis methods usually assume discrete equispaced
time and introductory textbooks on time series analysis either completely omit
the irregularly spaced case or they only dedicate
a very small section to continuous time models or to state-space models
with missing observations (\citeauthor{brockwell_time_1991}, \citeauthor{brockwell_introduction_2016},
\citeauthor{cryer_time_2008}, \citeauthor{chatfield_analysis_2003}).
The aim is thus to understand the problem of irregularly sampled time series data
and why standard time series method cannot deal with it.
Furthermore, the BP time series example should elucidate why and when Gaussian processes are a
suitable method to model a time series from which we only have observations sampled at
irregularly spaced time points.
Although the topic is motivated by a real dataset we will restrict ourselves to simulated data,
which will mimic the most important characteristics of BP time series data.
For simplicity reasons, the thesis deals only with systolic and never with diastolic
blood pressure.
Whenever blood pressure or BP is mentioned it refers to systolic blood pressure.


\section{Thesis Outline}

The aim is thus to understand the problem of irregularly sampled time series data
and why standard time series method cannot deal with it.
Furthermore, it should be elucidated why and when Gaussian processes are a
suitable method to model a time series from which we only have observations sampled at
irregularly spaced time points.










%%% Local Variables: 
%%% mode: latex
%%% TeX-master: "MasterThesisSfS"
%%% End: 

%! Author = gianna
%! Date = 06.04.23


\chapter{Characteristics of Time Series}

\section{Time Series Definition}\label{sec:time-series-definition}

A potentially unevenly spaced \textbf{time series} is a sequence of observation time and value pair $(t_i, x_i)$
    with strictly increasing observation times.
    Let $\mathbb{T}$ be a set of observation time points,
    then the sequence of random variables $(X_t: t \in \mathbb{T})$ or simply $(X_t)$ is a \textbf{time series process}
    with observation times $t \in \mathbb{T}$.
    More specifically:
\begin{itemize}
    \item $(X_t: t \in \{1, 2, \dots n\})$ refers to a discrete and equispaced time series of length n
    \item $(X_{t_i}: i \in \{1, 2, \dots n\})$ refers to an irregularly spaced time series of length n
    with observations at time points $t_1 < t_2 < \dots t_n$
    \item $(X_{t}: t \in (0, T])$ refers to a continuous time series
\end{itemize}

When $\mathbb{T}$ has finite length, we will often use a random column vector $\mathbf{X}$ to refer to
the time series process $(X_t)$.
Sometimes a time series model will be expressed as a random function $f: \mathbb{T} \to \mathbb{R}$ instead
of a collection of random variables.
Throughout the thesis, the term time series is used both to refer to the data $(x_t)$ and the process $(X_t)$ from which it is generated.

\section{Moments of a Time Series}\label{sec:time_series_moments}

A time series process $(X_t)$ is usually characterized by its first and second moment.

\begin{definition}(\citeauthor{brockwell_introduction_2016})\label{def:time_series_moments}
    The \textbf{mean function} of a time series $(X_t)$ is:
    \[
        \mu_X(t) = \ERW{X_t}
    \]
    The \textbf{covariance function} of a time series $(X_t)$ is:
    \[
        \gamma_X(r,s) = \COV{X_r, X_s} = \ERW{(X_r - \mu_X(r))(X_s-\mu_X(s))}
    \]
\end{definition}

\section{Stationarity}\label{sec:stationarity}

Given that one has only one observation $x_t$ per time point $t$,
a necessary condition to statistically learn from a time series is stationarity.

\begin{definition}(\citeauthor{brockwell_introduction_2016})
    A time series $(X_t)$ is strictly stationary iff
    the distribution of $(X_{t_1}, \dots X_{t_n})$ is identical to the distribution of
    $(X_{t_{1+h}}, \dots ,X_{t_{n+h}})$ for all $n \in \mathbb{N}^{+}$
    and shifts $h \in \mathbb{Z}$:
\end{definition}


\begin{definition}(\citeauthor{brockwell_introduction_2016})
    A time series $(X_t)$ is weakly stationary if
    \[
        \mu_X(t) \text{ is independent of t,}
    \]
    and
    \[
        \gamma_X(t+h, t) \text{ is indpendent of $t$ for each $h$}
    \]
\end{definition}

Whenever the term stationary is used, it is referring to weak stationarity.

\section{Special cases of Time Series Processes}\label{sec:example-of-time-series-processes}

\begin{example} If $(X_t)$ is a \textbf{white noise} process,
    then $X_t \sim WN(0, \sigma^2)$, that is $X_t \sim F$ iid for some
    distribution $F$ with mean $0$ and varaince $\sigma^2$. A special case is Gaussian White
    noise where $W_t \sim \N(0, \sigma^2)$ and $F= \Phi$
\end{example}

\begin{example}
    An equispaced time series process $(X_t: t \in \{1,2, \dots\})$ is called \textbf{autoregressive process}
    of order p or AR(p) if:
    \[
        X_t = \phi_1 X_{t-1} + \dots + \phi_p X_{t-p} + W_t
    \]
    where $\phi_p \neq 0$ and $(W_t)$ is a white noise process.
    The variable $W_t$ is called the innovation at time $t$ and is independent of all $X_k$, $k < t$.
\end{example}

\begin{example}
    An equispaced time series process $(X_t: t \in \{1,2, \dots\})$ is called \textbf{moving average process}
    of order q or MA(q) if:
    \[
        X_t = W_t + \theta_1 W_{t-1} + \dots \theta_q W_{t-q}
    \]
    where $\theta_q \neq 0$ and $(W_t)$ is a white noise process.
    The variable $W_t$ is called the innovation at time $t$ and is independent of all $X_k$, $k < t$.
\end{example}


\begin{example}
    An equispaced time series process $(X_t: t \in \{1,2, \dots\})$ is called \textbf{autoregressive moving average process}
    of autoregressive order p and moving average oder q or ARMA(p,q) if:
    \[
        X_t =  \phi_1 X_{t-1} + \dots + \phi_p X_{t-p} + \theta_1 W_{t-1} + \dots \theta_q W_{t-q} + W_t
    \]
    where $\phi_p \neq 0, \theta_q \neq 0$ and $(W_t)$ is a white noise process.
    The variable $W_t$ is called the innovation at time $t$ and is independent of all $X_k$, $k < t$

\end{example}




\section{Characteristics of the Blood Pressure Time Series and Observations}\label{sec:characteristics-of-the-blood-pressure-time-series}

This section sets the stage for simulating and modeling systolic blood pressure (BP).
First there one needs to make the distinction between the
true BP process and the BP measurements.
The former is usually unknown but is going to be simulated for the sake of this thesis.
The latter, are assumed to be noisy observations of this BP process.
We assume the following model for the BP measurement $Y(x)$
and the time series process $f(x)$, at some time point $x$:
\begin{align*}
    Y(x) = f(x) + \epsilon && \epsilon \sim \N(0, \sigma_n^{2})
\end{align*}
Note that both time series $f(x)$ and $Y(x)$ are described as random
functions. However, as mentioned in section \ref{sec:time-series-definition},
they could instead be described as collections of random variables with
$(Y_{t_i}: i \in \{1, 2, \dots n\})$ and $(f_{t_i}: i \in \{1, 2, \dots n\})$ with
$\epsilon_1 \dots \epsilon_n \iidsim \N(0, \sigma_n^{2})$.

The properties of the measurements $Y(x)$ are given by Aktiia's
user data, but for the underlying time series process $f(x)$ and the measurement noise
$\epsilon$, additional assumptions need to be made.

\subsection{Blood Pressure Time Series Process}\label{subsec:blood-pressure-time-series-process}
For the sake of this thesis, a BP time series process,
is expected to be the sum of the following components:
\begin{itemize}
    \item A seasonal component, representing the circadian cycle.
    BP is known to be higher during the day than during the night.
    \item An autoregressive component, since we assume the output variable to depend on its own previous values.
    \item A long term trend.
\end{itemize}


\subsection{Blood Pressure Measurements}\label{subsec:blood-pressure-measurements}
Based on the Aktiia user data, the following properties of
the blood pressure measurements have been identified:
\begin{enumerate}[\roman{enumi}.)]
    \item Measurements are irregularly spaced, i.e. the time between two consecutive measurements varies.
    \item Measurements are not uniformly sampled across time but the density of the observations should
    follow the circadian cycle (seasonal sampling).
    \item The sampling frequency varies from 0.5 to 4 measurements per hour.
    \item The difference between the average day and average night BP measurements is between
    0 and 20 mmHg and is on average 10 mmHg.
    \item The mean BP overall users is 120 mmHg.
    \item The within subject one-week sample variance is between 16 and
    144 mmHg\textsuperscript{2} and is on average 49 mmHg\textsuperscript{2}.
\end{enumerate}


\subsection{Measurement Noise}

The Aktiia measurements have been validated against some
reference method.
The measured variance of the differences between Atkiia measurements
and this reference is $62 mmHg^2$.
We can thus write:
\begin{align*}
    \Var(BP_{Ref} - BP_{Aktiia})
    & = 62 mmHg^2 = \Var(\epsilon_{Ref} - \epsilon_{Aktiia}) \\
    & = \Var(\epsilon_{Ref}) + \Var(\epsilon_{Aktiia}) - 2\Cov(\epsilon_{Ref},
    \epsilon_{Aktiia})
\end{align*}

If one further assumes that the noise variance of the reference method,
$\Var(\epsilon_{Ref})$, equals that of the Aktiia measurements, $\Var(\epsilon_{Aktiia})$,
and that $\Cov(\epsilon_{Ref}, \epsilon_{Aktiia})=0$, we obtain a noise variance
for the Aktiia measurements,
$\Var(\epsilon_{Aktiia})$, of 31 mmHg\textsuperscript{2}.




%! Author = gianna
%! Date = 06.04.23


\chapter{Time Series Decomposition and Regression}

%Many authors use the word trend only for a slowly changing mean func-
%tion, such as a linear time trend, and use the term seasonal component for a mean func-
%tion that varies cyclically.

\section{Introduction}

As most time series, the mean function of the BP time series is not constant in time and hence it is not stationary.
One can try to decompose the time series $Y(t)$ into a deterministic component, the mean function $\mu(t)$
and a zero mean stationary process $E(t)$:
\[ Y(t)= \mu(t) + E(t) \]

This decomposition allows to extract a stationary residual $E(t)$, for which we can find a probabilistic model
using the theory of such stationary time series processes. The idea is to then use this model in combination
with $\mu(t)$ to predict the BP value $Y(t)$ at any time $t$ including confidence intervals.

The task of time series decomposition is hence to estimate $\mu(t)$, which might be an arbitrary function of $t$, from the data.
Given our knowledge of the BP time series we will start with a simple linear model that
features a linear trend and a seasonal component with known frequency $f$.
If the seasonal component is represented by a cosine $\alpha \cos(2 \pi f t - \phi)$ we obtain the
following liner model for the BP time series $Y(t)$

\begin{gather*}
Y(t) = \beta_0 + \beta_1 t + \beta_2 \cos(2 \pi f t) + \beta_3 \sin(2 \pi f t) + E(t), \\
\text{where $\beta_2 = \alpha \cos(\phi)$ and $\beta_3 = \alpha \sin(\phi)$}
\end{gather*}

If we assume observations $Y_{t_i}$ at potentially unequally spaced
time points $t_i = t_1 \dots t_n$ we can write in matrix notation:
\begin{gather*}
\mathbf{Y} = X \beta + \mathbf{E}
\end{gather*}

Where $\mathbf{Y} = [Y_{t_1}, \dots Y_{t_n}]^{\top}$ is the vector of observations,
$X = [x_{t_1}, \dots x_{t_n}]^{\top}$ is the design matrix with i-th row
$x_{t_i} = [1, t_i, \cos(2 \pi f t_i), \sin(2 \pi f t_i)]^{\top}$
and $\mathbf{E} = [E_{t_1}, \dots E_{t_n}]^{\top}$ the values of the zero-mean stationary time series.

We can use ordinary least squares to find unbiased and asymptotically normal estimates $\hat{\beta}_{OLS} = (X^{\top}X) X^{\top}Y$
for the regression coefficients $\beta$ without the requirement of regularly spaced data points or uncorrelated errors
$E_{t_i}$ (\citeauthor{white_asymptotic_2001}).
In the case of uncorrelated errors $E_{t_i}$ with constant variance $\sigma^2$ we have
$Var(\mathbf{E}) = \sigma^2 I_n$ and an unbiased and consistent estimator for $\Psi = Var(\hat{\beta}_{OLS})$ is given by:
\begin{gather*}
\hat{\Psi} = \hat{\sigma}^2(X^{\top}X) \\
    \text{where $\hat{\sigma}^2=\frac{1}{n-p} \sum_{i = 1}^{n} (y_{t_i} - x_{t_i}^{\top} \hat{\beta}_{OLS})$ and $p=4$ in our example}
\end{gather*}

However, the assumption of uncorrelated errors $E_{t_i}$ is generally violated in the case of time series data and the
estimated covariance matrix $\hat{\Psi}$ is biased (\citeauthor{brockwell_introduction_2016}).



\section{Generalized Least Squares}

The argument presented in this section is based on the textbook of \citeauthor{brockwell_introduction_2016}.
If the covariance matrix of the errors $Var(\mathbf{E}) = \Sigma$ is known,
we can use generalized least squares to obtain a unbiased, consistent and efficient coefficient estimate:
\[\hat{\beta}_{GLS} = (X^{\top} \Sigma^{-1} X) X^{\top} \Sigma^{-1} Y\]
with unbiased and consistent covariance matrix estimate:
\[Var(\hat{\beta}_{GLS}) = (X^{\top} \Sigma^{-1} X)^{-1}\]

If $\Sigma$ is unknown one can exploit the knowledge we have about the stationary time series process $E(t)$ to estimate it.
Two approaches to estimate $\Sigma$, $\hat \beta $ and its covariance matrix will be presented in the following subsections.
Both methods assume an ARMA(p,q) process for $E(t)$ and equispaced time points. Hence we write $E(t)$ as $\{E_t\}$ and:

\begin{gather*}
    \Phi(B)E_t = \Theta(B)W_t, \text{where $W_t \sim WN(0, \sigma_w^2)$ and $t = 1, 2 \dots n$}
\end{gather*}


\subsection{Maximum-Likelihood Estimation}

If we additionally assume $W_t \sim N(0, \sigma_w^2)$, we can simultaneously estimate the regression coefficients and $\Sigma$ by
maximizing the Gaussian likelihood:

\begin{gather*}
    L(\beta, \phi, \theta, \sigma_w^2) = (2 \pi)^{-\frac{n}{2}} (det(\Sigma_n))^{-\frac{1}{2}} exp(-\frac{1}{2}
    (\mathbf{Y}-X\beta)^{\top} \Sigma_n^{-1}(\mathbf{Y}-X\beta))
\end{gather*}

Where the covariance matrix $\Sigma_n(\theta, \phi, \sigma_w^2)$ is parametrized by the ARMA coefficients $\theta, \phi, \sigma_w^2$,
based on the order of the ARMA processed assumed for $\{E_t\}$.


Assuming an ARMA(2,3) process for $\{E_t\}$ we can implement this approach in R using the nlme library (\citeauthor{box_time_1994})
:
\begin{verbatim}
    library(nlme)
    cs <- corARMA(from = ~t, p=2, q=3)
    fit.gls <- gls(y ~ t + cos(2 * pi * f * t) + sin(2 * pi * f * t), corr=cs)
\end{verbatim}


\subsection{Sandwich Estimation}
The second approach to fit an OLS regression first and correct the estimated covariance matrix $\Psi$ with a
sandwich estimator.
In the presence of autocorrelation one usually estimates $\Phi = \frac{1}{n} X^{\top} \Sigma X$,
the covariance matrix of the score functions
$V_i(\beta) = x_{t_i}(y_{t_i} - x_{t_i}^{\top}\beta)$, which can then be used to derive $\Psi$:

\begin{equation}\label{eq:sandwich}
\Psi = Var(\hat \beta_{OLS}) = (X^{\top} X)^{-1} X^{\top} \Sigma X (X^{\top}X)^{-1} =
(\frac{1}{n} X^{\top} X)^{-1} \frac{1}{n} \Phi (\frac{1}{n} X^{\top} X)^{-1}
\end{equation}

The general form of the estimators for $\Phi$ is:

\begin{gather*}
\hat{\Phi} = \frac{1}{n} \sum_{i,j=1}^{n} w_{|i-j|}\hat{V_i}\hat{V_j}^{\top}
\end{gather*}
where $w=[w_0, \dots w_{n-1}]^{\top}$ is a weight vector.

$\Phi$ can then be plugged into the equation \ref{eq:sandwich} to obtain a
heteroskedasticity and autocorrelation consistent (HAC) covariance estimate $\hat{\Psi}_{HAC}$

\citeauthor{newey_automatic_1994}, \citeauthor{andrews_heteroskedasticity_1991} and others have suggested different approaches
for calculating the weights $w$. They all yield decreasing weights with increasing lag $l=|i-j|$.
The R sandwich package implements some of these methods to estimate $\hat{\Psi}_{HAC}$.
An introduction to the sandwich package and how it can be used
for inference is described by \citeauthor{zeileis_econometric_2004}.


\subsection{Extension to Irregularly Spaced Time Series}

Although literature and "ready to use" implementations only seem to exist for the equispaced case,
both of the approaches described above could probably be extended to the case of irregularly spaced time series.
For the Maximum-Likelihood approach one would need to adapt the parametrization of the covariance matrix $\Sigma_n$, where
the covariance at different time points would need to depend on their actual time difference rather than the lag.
A similar adaption could be done for the sandwich estimator, where the weights would need to depend on the time
difference rather than on the lag.



\chapter{First Chapter}

\section{To include a picture}
\begin{figure}[hbt!]%--- Picture 'H'ere, 'B'ottom or 'T'op; '!' Try to
                    %impose your will to LaTeX
  \epsfCfile{.85}{geys-2kern} %<< no file extension
  %%         --- .85 stands for 85% of text width
  \caption[Geyser data: binned histogram, Silverman's and another
  kernel]%<<-- Legend for the list of figures at the beginning of you thesis
  {Old Faithful Geyser eruption lengths, $n=272$; binned data and two
    (Gaussian) kernel density estimates ($\times 10$) with $h=h^*= .3348$
    and $h= .1$ (dotted).}% legend displayed below the graph.
  \label{fig:geys1}
\end{figure}

Or also with \texttt{includegraphics}:
\begin{figure}[hbt!]%--- Picture 'H'ere, 'B'ottom or 'T'op; '!' Try to
                    %impose your will to LaTeX
  \centering
  \includegraphics[width=.5\textwidth]{geys-2kern} %<< no file extension
  %%         --- .5\textwidth stands for 50% of text width
  \caption[Geyser data: binned histogram, Silverman's and another
  kernel]%<<-- Legend for the list of figures at the beginning of you thesis
  {Old Faithful Geyser eruption lengths, $n=272$; binned data and two
    (Gaussian) kernel density estimates ($\times 10$) with $h=h^*= .3348$
    and $h= .1$ (dotted).}% legend displayed below the graph.
  \label{fig:geys2}
\end{figure}

\section{To make a proof}
\begin{proof}
  $1 + 1 = 2$
\end{proof}

\section{To include \Rp code}
See information in Appendix~\ref{app:complement}.


\section{Other information}
Put a text between quotes: make sure to use nice quotes, such as ``quote''.

Cite an article or book you refer shortly here, and then listed in the bibliography.
%%--> in file   myReferences.bib  (same directory)
Or mention that \citeauthor{robinson_estimation_1977}  (a person) (two
persons) have already done quite a bit work.

\citeauthor{marvasti_nonuniform_2001}


Referencing a different part of your work: please refer to Appendix \ref{app:complement}.


%%% Local Variables: 
%%% mode: latex
%%% TeX-master: "MasterThesisSfS"
%%% End: 


%%\include{Chapter...}
\include{Summary} 

%%%%%%%%%%%%%%%%%%%%%%%%%%%%%%%%%%%%%%%%%%%%%%%%%
%%% Bibliography                              %%%
%%%%%%%%%%%%%%%%%%%%%%%%%%%%%%%%%%%%%%%%%%%%%%%%%
\addtocontents{toc}{\vspace{.5\baselineskip}}
\cleardoublepage
\phantomsection
\addcontentsline{toc}{chapter}{\protect\numberline{}{Bibliography}}
\bibliography{master_thesis_gm}
%% All books from our library (SfS) are already in a BiBTeX file
%% 'Assbib.bib' (included here as well), using
% \bibliography{myReferences,Assbib}
% ---------------------------------- instead of the above



%%%%%%%%%%%%%%%%%%%%%%%%%%%%%%%%%%%%%%%%%%%%%%%%% 
%%% Appendices (if needed, e.g. for R code)   %%%
%%%%%%%%%%%%%%%%%%%%%%%%%%%%%%%%%%%%%%%%%%%%%%%%%
\addtocontents{toc}{\vspace{.5\baselineskip}}
\appendix
\chapter{Complementary information}\label{app:complement}


Additional material. For example long mathematical derivations could be
given in the appendix. Or you could include part of your code that is
needed in printed form. You can add several Appendices to your thesis (as
you can include several chapters in the main part of your work).

\section{Ornstein-Uhlenbeck Process}\label{app:ou}

The autocovariance function of an Ornstein-Uhlenbeck process can be derived by solving the stochastic differential equation (SDE) that defines the process.

Starting with the SDE for an OU process:

$$dX_t = \theta (\mu - X_t)dt + \sigma_w dW_t,$$

where $X_t$ is the value of the process at time $t$, $\theta$ is a positive constant that determines the speed of mean reversion,
$\mu$ is the long-term mean of the process, $\sigma_w$ is the standard deviation of the random shocks, and $W_t$ is a standard Wiener process or Brownian motion.

The solution to the SDE is:

$$ X_t = X_0 e^{-\theta t} + \mu (1-e^{-\theta t}) +
\sigma_w e^{-\theta t} \int_{0}^{t} e^{\theta s} dW_s$$

The process is stationary if $\theta > 0$.
The autocovariance function of an OU process is given by
$Cov(X_t, X_{t-k}) = \frac{\sigma_w^2}{2\theta} e^{-\theta k}$,
where $k\geq 0$ and $\theta > 0$.

This is the same expression as we have obtained in \ref{kernel-matern-ar1}, where
$k(0) = \sigma^2 = \frac{\sigma_w^2}{2\theta}$ and $l=1/\theta$

To see how the Ornstein-Uhlenbeck can be considered a continuous time analogue to the discrete time
AR(1) process one can use the Euler-Maryuama discretization of the process.
Considering again the SDE for an OU process:
$$dX_t = \theta (\mu - X_t)dt + \sigma_w dW_t,$$
The process can be discretized at times $(k \Delta t)_{k \in \mathbb{N}_0}$:

$$ X_{k+1} - X_k = \theta \mu \delta t - \theta X_k \Delta t + \sigma_w (W_{k+1} - W_k)$$

The random variables $(W_{k+1} - W_k)$ are independent and identically distributed normal random variables
with expected value zero and variance $\Delta t$.
Therefore, we can set $\sigma_w (W_{k+1} - W_k) = \sigma_w \sqrt{\Delta t} \epsilon$ with $\epsilon \sim \N(0,1)$
to obtain the following recursion:
$$ X_{k+1} = \theta \mu \Delta t - (\theta \Delta t - 1) X_k + \sigma_w \sqrt{\Delta t} \epsilon$$

The recursion for an AR(1) process is:
$$ X_{k+1} = c + a X_k + b \epsilon$$
Which is identical to the expression above if $c= \theta \mu \Delta t$, $a=1- \theta \Delta t$ and
$b= \sigma_w \sqrt{\Delta t}$



\section{Properties of the Simulated Time Series Samples}\label{sec:properties-of-the-simulated-time-series-samples}

This section presents the distribution of some crucial properties
from the simulated BP time series.
These histograms have been created by drawing 100 samples from the true GP.

The shown property distributions should match those from Section \ref{sec:characteristics-of-the-blood-pressure-time-series}.

\begin{figure}[h!]
    \centering
    \includegraphics[width=0.6\linewidth]{
        Pictures/variance_distribtution/sin_rbf_seasonal_default/09_07_07_12_53/variance_y_true_train_summary.pdf}
    \caption{The one-week sample variance should span from from 16 to 144 mmHg², with an average of 49 mmHg²}
    \label{fig:variance}
\end{figure}

\begin{figure}[h!]
    \centering
    \includegraphics[width=0.6\linewidth]{
       Pictures/variance_distribtution/sin_rbf_seasonal_default/09_07_07_12_53/variance_dip_ampl_summary}
    \caption{Half of the difference between average daytime and nighttime BP measurements. Should fall
    within the range of 0 to 10 mmHg}
    \label{fig:dip_ampl}
\end{figure}



\begin{figure}[h!]
    \centering
    \includegraphics[width=0.6\linewidth]{
       Pictures/variance_distribtution/sin_rbf_seasonal_default/09_07_07_12_53/variance_2.24**2 * Matern(length_scale=3, nu=0.5) * 1**2_summary.pdf}
    \caption{The variance of the AR(1) component. There exists no target values for this.}
    \label{fig:var_matern}
\end{figure}

\begin{figure}[h!]
    \centering
    \includegraphics[width=0.6\linewidth]{
       Pictures/variance_distribtution/sin_rbf_seasonal_default/09_07_07_12_53/variance_2.24**2 * RBF(length_scale=50) * 1**2_summary.pdf}
    \caption{The variance of the RBF component. There exists no target values for this.}
    \label{fig:var_rbf}
\end{figure}


\begin{figure}[h!]
    \centering
    \includegraphics[width=0.6\linewidth]{
       Pictures/variance_distribtution/sin_rbf_seasonal_default/09_07_07_12_53/variance_14**2 * ExpSineSquared(length_scale=3, periodicity=24) * 1**2_summary.pdf}
    \caption{The variance of the periodic component.}
    \label{fig:var_periodc}
\end{figure}



%
%\section{Including \Rp code with verbatim}
%A simple (rather too simple, see~\ref{App:listings}) way to include code or
%{\it R} output is to use
%\texttt{verbatim}. It just prints the text however it is (including all
%spaces, ``strange'' symbols,...) in a slightly different font.
%\begin{verbatim}
%## loading packages
%library(RBGL)
%library(Rgraphviz)
%library(boot)
%
%## global variables
%X_MAX <- 150
%
%   This allows me to put as many s  p a   c es   as I want.
%I can also use \ and ` and & and all the rest that is usually only
%accepted in the math mode.
%
%I can also make as
%                  many
%             line
%    breaks as
%I want... and
%             where I want.
%\end{verbatim}
%
%But really recommended,  much better is the following:
%
%\section{Including \Rp code with the \emph{listings} package}\label{App:listings}
%However, it is much nicer to use the \emph{listings} package to include \Rp
%code in your report. It allows you to number the lines, color the comments
%differently than the code, and so on.
%All the following is produced by simply writing
%\verb! \lstinputlisting{Pictures/picture.R} !  in your \LaTeX\ ``code'':
%
%\lstinputlisting{Pictures/picture.R}
%
%or \verb!\lstinputlisting{/u/maechler/R/Pkgs/sfsmisc/R/ellipse.R}! :
%
%\lstinputlisting{ellipse.R}% was /u/maechler/R/Pkgs/sfsmisc/R/ellipse.R
%
%\section{Using \texttt{Sweave} (or \texttt{knitr}) to include \Rp code (and more) in your report}
%The easiest (and most elegant) way to include \Rp code and its output (and
%have all your figures up to date with your report) is to use Sweave---or the
%\href{https://cran.R-project.org/package=knitr}{\texttt{knitr}} R package with even more possibilities.
%% You can find an introduction Sweave in \texttt{/u/sfs/StatSoftDoc/Sweave/Sweave-tutorial.pdf}.
%
%Search the web to find lots of intro material on how to use Sweave or
%\href{https://en.wikipedia.org/wiki/Knitr}{knitr (on Wikipedia)}.

%%% Local Variables: 
%%% mode: latex
%%% TeX-master: "MasterThesisSfS"
%%% End: 

\include{Appendix2}
\chapter{2nd Appendix: More sophisticated R code listing} \label{appendix-more-R}

Chapter-wise listing of parts of R code, using
\begin{itemize}
\item \texttt{firstline=n1}
\item \texttt{lastline=n2}
\item \texttt{title=<text>}
\end{itemize}
e.g., for the first example below
\begin{verbatim}
\lstinputlisting[firstline=1,lastline=20,
                 title= \texttt{ellipse.R}]{ellipse.R}
\end{verbatim}
and the second example
\begin{verbatim}
\lstinputlisting[firstline=20,lastline=40,
             title=\texttt{ellipse.R}]{ellipse.R}
\end{verbatim}
% \section{Chapter 2} \label{app 2}

% \lstinputlisting[firstline=1,lastline=77,
% title=\texttt{analytic\_efficiency.R}]{../RCode/analytic_efficiency.R}
% %\lstinputlisting[firstline=,lastline=]{../RCode/???.R}

\bigskip% or even  \clearpage

%-----------------------------------------------------------------------------------------
\section{Chapter 5} \label{app 5}

% \lstinputlisting[firstline=1,lastline=71,
%                  title=\texttt{loss-fn\_rotated.R}]{../RCode/loss-fn_rotated.R}
\lstinputlisting[firstline=1,lastline=20,
                 title= \texttt{ellipse.R}]{ellipse.R}

\medskip
                 
\lstinputlisting[firstline=20,lastline=40,
                 title=\texttt{ellipse.R}]{ellipse.R}
%\lstinputlisting[firstline=,lastline=]{../RCode/???.R}
%\lstinputlisting[firstline=,lastline=]{../RCode/???.R}

% \clearpage
%-----------------------------------------------------------------------------------------
% \section{Chapter 7} \label{app 7}

% \lstinputlisting[firstline=1,lastline=35,
%                  title= \texttt{stat.test} from \texttt{lmrob2-fn.R}]{../RCode/lmrob2-fn.R}
% \lstinputlisting[firstline=41,lastline=194,
%                  title=\texttt{M.optimal.ms} from \texttt{lmrob2-fn.R}]{../RCode/lmrob2-fn.R}
%\lstinputlisting[firstline=,lastline=]{../RCode/???.R}
%-----------------------------------------------------------------------------------------

%%% Local Variables:
%%% mode: latex
%%% TeX-master: "MasterThesisSfS"
%%% End:



%% Epilogue (optional)
\addtocontents{toc}{\vspace{.5\baselineskip}}
\cleardoublepage
\phantomsection
\addcontentsline{toc}{chapter}{\protect\numberline{}{Epilogue}}
\markboth{Epilogue}{Epilogue}
\include{Epilogue}


%%%%%%%%%%%%%%%%%%%%%%%%%%%%%%%%%%%%%%%%%%%%%%%%%% 
%%% Declaration of originality (Do not remove!)%%%
%%%%%%%%%%%%%%%%%%%%%%%%%%%%%%%%%%%%%%%%%%%%%%%%%%
%% Instructions:
%% -------------
%% fill in the empty document confirmation-originality.pdf electronically
%% print it out and sign it
%% scan it in again and save the scan in this directory with name
%% confirmation-originality-scan.pdf 
%%
%% General info on plagiarism:
%% https://www.ethz.ch/students/en/studies/performance-assessments/plagiarism.html 
\cleardoublepage
\includepdf[pages={-}, frame=true,scale=1]{confirmation-originality-scan.pdf}
\end{document}

%%% Local Variables:
%%% mode: latex
%%% TeX-master: "MasterThesisSfS"
%%% End:
