%! Author = gianna
%! Date = 06.04.23


\chapter{Characteristics of Time Series}

A \textbf{time series} is a set of observations $x_t$ , each one being recorded at a specific time $t \in T_0$.
$T_0$ is the set of times at which observations are made.
In case of discrete equispaced time series this is a discrete set, e.g. $T_0 = {1, 2, \dots T}$
and for continuous time series it is an interval, e.g. $T_0 = [1, T]$.

A \textbf{time series model} for the observed data ${x_t}$ is specified by the collection of random variables
${X_t}$ of which ${x_t}$ is thought to be a realization.
Alternatively the time series model can also be considered a random function $f: T_0 \to \mathbb{R}$

Throughout the thesis the term time series is used both refer to the data and the process from which it is generated.

%a specification of the joint
%distributions (or possibly only the means and covariances) of a sequence of random
%variables ${X_t}$ of which ${x_t}$ is postulated to be a realization.

\textbf{mean function}

\textbf{autocovariance function}


\section{Stationarity}
Stationarity is needed for being able to statistically learn from time series data.



\section{ARMA Model}

Autoregressive Process
Moving Average Process



\section{Characteristics of the Blood Pressure Time Series}

circadian cycle

