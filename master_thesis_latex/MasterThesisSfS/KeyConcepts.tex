%! Author = gianna
%! Date = 06.04.23


\chapter{Characteristics of Time Series}

A \textbf{time series} $(x_t: t \in T_0t)$ is a collection of observations $x_t$ ,
each one being recorded at a specific time $t$. $T_0$ is the set of times at which observations are made.
In case of discrete time series $T_0$ is a discrete set, e.g. for the equispaced case $T_0 = \{1, 2, \dots T\}$ and
for the unequally spaced case $T_0 = \{t_1, t_2, \dots t_n\}$ with $t_1 < t_2 < \dots t_n$.
For continuous time series $T_0$ is an interval, e.g. $T_0 = (0, T]$.

A \textbf{time series model} for the observed data $(x_t: t \in T_0)$ is specified by the collection of random variables
$(X_t: t \in T_0)$ of which $(x_t: t \in T_0)$ is thought to be a realization.
Alternatively the time series model can also be considered a random function $f: T_0 \to \mathbb{R}$.

Throughout the thesis the term time series is used both refer to the data and the process from which it is generated.

\citeauthor{brockwell_introduction_2016}


# TODO Notation should be adapted/extended to unequally spaced case.


\textbf{mean function}

\textbf{autocovariance function}


\section{Stationarity}
Stationarity is needed for being able to statistically learn from time series data.



\section{ARMA Model}

Autoregressive Process
Moving Average Process



\section{Characteristics of the Blood Pressure Time Series}

circadian cycle






