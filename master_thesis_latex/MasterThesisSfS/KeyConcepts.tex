%! Author = gianna
%! Date = 06.04.23


\chapter{Characteristics of Time Series}

\section{Time Series Definition}

    A potentially unevenly spaced \textbf{time series} is a sequence of observation time and value pair $(t_i, x_i)$
    with strictly increasing observation times.
    Let $\mathbb{T}$ be a set of observation time points,
    then the sequence of random variables $(X_t: t \in \mathbb{T})$ or simply $(X_t)$ is a \textbf{time series process}
    with observation times $t \in \mathbb{T}$.
    More specifically:
\begin{itemize}
    \item $(X_t: t \in \{1, 2, \dots n\})$ refers to a discrete and equispaced time series of length n
    \item $(X_{t_i}: i \in \{1, 2, \dots n\})$ refers to an irregularly spaced time series of length n
    with observations at time points $t_1 < t_2 < \dots t_n$
    \item $(X_{t}: t \in (0, T])$ refers to a continuous time series
\end{itemize}

When $\mathbb{T}$ has finite length, we will often use a random column vector $\mathbf{X}$ to refer to
the time series process $(X_t)$.
Sometimes a time series model will be expressed as a random function $f: \mathbb{T} \to \mathbb{R}$ instead
of a collection of random variables.
Throughout the thesis, the term time series is used both to refer to the data $(x_t)$ and the process $(X_t)$ from which it is generated.

\section{Moments of a Time Series}\label{sec:time_series_moments}

A time series process $(X_t)$ is usually characterized by its first and second moment.

\begin{definition}(\citeauthor{brockwell_introduction_2016})\label{def:time_series_moments}
    The \textbf{mean function} of a time series $(X_t)$ is:
    \[
        \mu_X(t) = \ERW{X_t}
    \]
    The \textbf{covariance function} of a time series $(X_t)$ is:
    \[
        \gamma_X(r,s) = \COV{X_r, X_s} = \ERW{(X_r - \mu_X(r))(X_s-\mu_X(s))}
    \]
\end{definition}

\section{Stationarity}\label{sec:stationarity}

Given that one has only one observation $x_t$ per time point $t$,
a necessary condition to statistically learn from a time series is stationarity.

\begin{definition}(\citeauthor{brockwell_introduction_2016})
    A time series $(X_t)$ is strictly stationary iff
    the distribution of $(X_{t_1}, \dots X_{t_n})$ is identical to the distribution of
    $(X_{t_{1+h}}, \dots ,X_{t_{n+h}})$ for all $n \in \mathbb{N}^{+}$
    and shifts $h \in \mathbb{Z}$:
\end{definition}


\begin{definition}(\citeauthor{brockwell_introduction_2016})
    A time series $(X_t)$ is weakly stationary if
    \[
        \mu_X(t) \text{ is independent of t,}
    \]
    and
    \[
        \gamma_X(t+h, t) \text{ is indpendent of $t$ for each $h$}
    \]
\end{definition}

Whenever the term stationary is used, it is referring to weak stationarity.

\section{Special cases of Time Series Processes}\label{sec:example-of-time-series-processes}

\begin{example} If $(X_t)$ is a \textbf{white noise} process,
    then $X_t \sim WN(0, \sigma^2)$, that is $X_t \sim F$ iid for some
    distribution $F$ with mean $0$ and varaince $\sigma^2$. A special case is Gaussian White
    noise where $W_t \sim \N(0, \sigma^2)$ and $F= \Phi$
\end{example}

\begin{example}
    An equispaced time series process $(X_t: t \in \{1,2, \dots\})$ is called \textbf{autoregressive process}
    of order p or AR(p) if:
    \[
        X_t = \phi_1 X_{t-1} + \dots + \phi_p X_{t-p} + W_t
    \]
    where $\phi_p \neq 0$ and $(W_t)$ is a white noise process.
    The variable $W_t$ is called the innovation at time $t$ and is independent of all $X_k$, $k < t$.
\end{example}

\begin{example}
    An equispaced time series process $(X_t: t \in \{1,2, \dots\})$ is called \textbf{moving average process}
    of order q or MA(q) if:
    \[
        X_t = W_t + \theta_1 W_{t-1} + \dots \theta_q W_{t-q}
    \]
    where $\theta_q \neq 0$ and $(W_t)$ is a white noise process.
    The variable $W_t$ is called the innovation at time $t$ and is independent of all $X_k$, $k < t$.
\end{example}


\begin{example}
    An equispaced time series process $(X_t: t \in \{1,2, \dots\})$ is called \textbf{autoregressive moving average process}
    of autoregressive order p and moving average oder q or ARMA(p,q) if:
    \[
        X_t =  \phi_1 X_{t-1} + \dots + \phi_p X_{t-p} + \theta_1 W_{t-1} + \dots \theta_q W_{t-q} + W_t
    \]
    where $\phi_p \neq 0, \theta_q \neq 0$ and $(W_t)$ is a white noise process.
    The variable $W_t$ is called the innovation at time $t$ and is independent of all $X_k$, $k < t$

\end{example}




\section{Characteristics of the Blood Pressure Time Series}
TODO
circadian cycle

