\chapter{Discussion and Conclusion}\label{ch:discussion-and-conclusion}

\subsection{Comparison of GP Regression and Baseline Methods}

Overall, when considering different downsampling patterns and target measures,
GP regression outperforms the baseline methods. This superiority is particularly
evident when calculating the mean over small time windows, such as one-hour and one-day means.
This performance can be attributed to the fact that GP regression explicitly models
the dependencies among BP values across different time points. Consequently,
the uncertainty predictions are based on the amount of data available at time
points that are highly correlated, either positively or negatively, with the time
point of prediction.
Since the degree of correlation depends on the proximity to the prediction point,
this results in larger CI when data density is low around the prediction point.

Linear regression, on the other hand, provides the narrowest CIs and maintains adequate
CI coverage for the one-week mean under large downsampling factors. However,
linear regression does not exhibit significant improvement with an increase in data.
This can be attributed to the inherent
limitations of the linear model, which can only capture a linear trend with a perfect sinusoidal
seasonal pattern. This characteristic makes the method less reliant on the amount of data available.

In contrast, spline regression is a non-parametric method and is thus more data-dependent.
Spline regression does not yield meaningful results under large downsampling factors,
except when estimating TTR, where the locally unstable BP estimates have a relatively smaller impact.
For the same reason, spline regression does not attempt to fit a cyclic pattern and therefore struggles with seasonal sampling.

Although GPs fall under the category of non-parametric methods, they offer the option to express
prior beliefs about the function of interest through the choice of kernel.
In our case, a function with a cyclic pattern with a periodicity of 24 hours, an AR component, and a long-term trend are favored.
This choice does not impose as many constraints on the predictions as linear regression does.
These properties make GP regression the ideal candidate for analyzing BP time series based on irregularly spaced samples.


Seasonal sampling leads, in all cases, to reduced performance.
Interestingly, in the face of seasonal sampling, more data generally leads to
a reduction in Ci coverage and width for linear regression,
whereas for GP regression, CI coverage would increase,
and for spline regression, it would remain at the same level.



\section{Limitations and Future Work}

In the current study GP regression is used to estimate values that
where generated from a GP itself. Not only that but also the same
combination of kernel types, i.e. RBF, Matérn and Periodic kernel,
have been used for simulation and estimation.
When fitting a GP, only the kernl hyperparameters had to be found.
Although this seemed to be already an adequate challenge, it
would be interesting to investigate the impact of using a misspecified
kernel on estimation time.
In addition to that it would be interesting to see how the predictions
would be affected by non Gaussianities added to the simulated values.
This could be achieved by simply adding non Gaussian noise to the
true BP values.
Apart from that one should probably also try to come up with
a completely different way of simulating BP values, ideally also
providing more control over the simulated samples. Since when drawing from
a GP at random, the only control over the shape of functions produced is
through the chosen kernel function.

Although the characteristics of the simulated BP values have been compared
to those that are observed in real world, the level of measurement error,
of the periodic and AR component remain still largely
unknown. Adversarial analysis could further be expanded by investigating
the effect of varying relative contributions of the different kernels and the measurement noise
on the prediction performance.
It has been shown that a larger periodic component in the singal,
makes predictions easier while for obvious reasons that would also be the case
if the simulated measurement noise was reduced. Hence by varying the contribution
of the different components the limits of the regression methods should be
identified.

GP regression credible interval estimtes have been calculated based on the
equal-tailed credible interval (ETI). Another common credible interval is the
Highest posterior density interval (HDI), which leads to different intervals
mainly in the face of asymmetric distributions, which might be epxected
for TTR. So for the nest simulation study, one should extract both HDI and ETI
to understand which one is more suitable for the problem at hand.


Of particular interest to the company are further:
\begin{itemize}
    \item Simulation of a seasonal component that would evolve over time. This
    could be simply achieved by multiplying the Periodic kernel, that was
    so far used for simulation, with another kernel that models this evolution, e.g. an RBF kernel.
    \item Calculation of day and night BP. This presumes a certain definition of day and night
    which could be obtained with the aid of the predicted cyclic component.
    \item The computational complexity of the different regression methods.
\end{itemize}









%The method used for spline regression should be fine-tuned.
%One could use ridge regression on the
%generated B-spline basis functions instead of OLS regression. In this case
%one would identify the penalty parameter $\lambda$ through cross validation and
%choose a fixed high enough number of knots.
%Instead of
%using OLS regression on the B-spline bases generated form the input times,
%choosing the number of knots of the splines through cross validation
%it is probably more common to generate B-spline bases and
%
%
%
%\begin{itemize}
%    \item Use other methods than GPs to simulate BP time series
%    \item Investigate the impact of misspecified kernel functions and the effect of adding npn-gaussian errors.
%    \item Simulate evolving seasonal component by multiply the periodic kernel with e.g. an RBF kernel
%    \item Since measurement error, the periodic and AR component are largely unknown,
%    investigate the impact of varying the relative contributions of the different kernels.
%    \item Use highest posterior density credible interval (HDI) instead of the equal-tailed interval.
%    \item Define day and night BP values with the aid of the predicted cyclic component
%    \item Investigate computational complexity of GP and baseline methods
%    \item Highest posterior density interval (HDI) (https://easystats.github.io/bayestestR/reference/hdi.html)
%    Equal-tailed interval (ETI) (https://easystats.github.io/bayestestR/reference/eti.html)
%
%\end{itemize}

