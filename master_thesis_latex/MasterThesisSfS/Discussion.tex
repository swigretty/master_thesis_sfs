\chapter{Summary}\label{ch:discussion-and-conclusion}

\subsection{Comparison of GP Regression and Baseline Methods}

Overall downsampling patterns and target measures, GP regression performs best.
This is especially the case for the mean calculated over small time windows, i.e. one-hour and
one-day mean. This can be explained by the fact, that GP regression explicitly
models the dependencies of BP values between time points.
Hence uncertainty predictions
are based on how much data there is at time points which highly (negatively or positively)
correlated with the time point of prediction. Since correlation is depends on
the proximity to the point of prediction, this leads to larger CIs in case
of low data desity around the point of prediction.
Linear regression does provide the narrowest CIs with adequate CI coverage for the
one-week mean under large downsampling factors.
However generally linear regression does not improve a lot with more data, escpecially
in the presence of seasonal sampling.
This can be explained by the fact, that the linear model used is very constrained and
can only fit a linear trend with a perfect sinusoidal seasonal pattern. This
make the method lesss data dependent.
Smoothing spline regerssion  on the other
hand does not produce valuable results under large downsampling factors
except when estimating TTR, where the locally unstable BP estimates does not
have a huge impact. This behavior of s




\section{Limitations and Future Work}


    Highest posterior density interval (HDI) (https://easystats.github.io/bayestestR/reference/hdi.html)
    Equal-tailed interval (ETI) (https://easystats.github.io/bayestestR/reference/eti.html)


Investigate computational complexity

\begin{itemize}
    \item Use other methods than GPs to simulate BP time series
    \item Investigate the impact of misspecified kernel functions and the effect of adding npn-gaussian errors.
    \item Simulate evolving seasonal component by multiply the periodic kernel with e.g. an RBF kernel
    \item Since measurement error, the periodic and AR component are largely unknown,
    investigate the impact of varying the relative contributions of the different kernels.
    \item Use highest posterior density credible interval (HDI) instead of the equal-tailed interval.
    \item Define day and night BP values with the aid of the predicted cyclic component
    \item Investigate computational complexity of GP and baseline methods

\end{itemize}

