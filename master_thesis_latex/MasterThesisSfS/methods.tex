\usepackage{amsmath}\chapter{Methods}\label{ch:methods}

The last chapter introduced Gaussian process regression
to establish a mapping between a time point x and its
corresponding BP value.
Since Gaussian Processes are capable of modeling time series in continuous
time and hence deal with irregularly spaced data,
they seem to be a good candidate for modeling a time series from which
we only have irregularly sampled noisy measurement.
To identify how suited GPs are for the problem at hand, the plan is to:
\begin{itemize}
    \item Simulate a BP time series and some noisy measurements of the time series
    \item Estimate some target measures from these measurements.
    \item Compare the aptness of GP regression to estimate the target measures to that of some baseline methods.
    \item Investigate the impact of the sampling pattern on the target measure estimates, e.g
    How densely does the data need to be sampled to get good estimates?
    What happens if the data is not uniformly sampled across time ?
\end{itemize}

The next section will provide you with an overview of the steps involved
in quantifying the performance of a given method.


\section{Overview}

This is a list of the steps involved in assessing the performance of a given
method. More information can be found in the referenced sections:

\begin{enumerate}
    \item Simulate true BP time series at high resolution (section \ref{sec:blood-pressure-time-series-simulation})
    \item Simulate BP measurement by subsampling the true time series and adding noise (section \ref{sec:adversarial-analysis})
    \item Fit Regression (For GP Regression see section \ref{sec:gaussian-process-regression}
    for baseline methods see \ref{sec:baseline-methods})
    \item Extract target measure including CI (section \ref{sec:target-measures})
    \item Calculate performance metric between true and predicted values (section \ref{sec:performance-metric})
\end{enumerate}


\begin{algorithm} \caption{Simulation and Evaluation Flow}
 \hspace*{\algorithmicindent} \textbf{Input:} $t$, SamplingScheme, $GP_{true}$,
RegressionMethod, TargetMeasure\\
 \hspace*{\algorithmicindent} \textbf{Output:} CiCoverage, CiWidth
\begin{algorithmic}[1]
    \For {$i \gets 1$ to $N$}

        \State $f \gets$ Sample from $GP_{true}$
        \State $y \gets f + \epsilon$, $\epsilon_i \text{ i.i.d.} \sim \N(0, \sigma_n^2)$
        \State $y_{train} \gets$ subsample $y$ based on sampling scheme
        \State RegressionMethod.fit($y_{train} $)
        \State $\hat{y} \gets$ RegressionMethod.predict($t$)
        \State ci coverage $\gets$ vla
        \EndFor
    \Ensure $V \approx V^\pi$
\end{algorithmic}
\end{algorithm}


\section{Simulation and Evaluation Flow GP}

Simulation of true BP signal and measurements:
    \begin{itemize}
        \item $X=\{x_1, \dots x_n\}$: The input time points of interest, i.e. 1 week with 10 BP values per hour.
        \item $f_X := \{f(x_1) \dots f(x_n)\}$: The true BP signal $f(x)$ drawn from $GP(0, k_{true}(x,x'))$ evaluated at inputs $X$.
        \item $y_X := \{y(x_1) \dots y(x_n)\}$: The noisy BP measurements. $y(x)= f(x) + \epsilon$ with $\epsilon \sim N(0, \sigma_n^2)$
    \end{itemize}

Based on subsampling scheme choose $X_{train} \subset X$ with $|X_{train}| = m$ and $X_{test} = X \setminus X_{train}$.
\begin{itemize}
    \item $y_{train} := \{y(x_i) | x_i \in X_{train}\}$
    \item  $f_{train} := \{f(x_i) | x_i \in X_{train}\}$
\end{itemize}


	Fit GP regression model to training data $y_{train}$, $X_{train}$:
    \begin{itemize}
        \item $k_{fit}$: The fitted kernel function with hyperparameters $\theta$ that maximizes the marginal likelihood
        $p(y_{train}| \theta)$
        \item $p(f_X| y_{train}, k_{fit}) = N(\bar{\mu}, \bar{\Sigma})$:
        The posterior (predictive) probability density of $f_X$. The posterior mean vector
        $\bar{\mu} \in \mathbb{R}^n$ contains the point estimates for $f_X$.
        \begin{itemize}
            \item $\bar{\mu}_{train} := \{\bar{\mu}_i | x_i \in X_{train}\}$
            \item $\bar{\Sigma}_{train} := \bar{\Sigma}_{i,j}$ where $\{i | x_i \in X_{train}\}$ and $\{j | x_j \in X_{train}\}$

        \end{itemize}
    \end{itemize}

	Output of GP Regression are predictive probabilities $p(f_{train}| y_{train})$,
		$p(f_{test}| y_{train})$ and $p(f_{X}| y_{train})$:
			\begin{itemize}
				\item Note $\bar{\mu}_{train} \neq f_{train}$ due to the measurement error. \\ If
				$f_{train}$ was known: $p(f_{train}| f_{train}, GP_{fit}) = N(\bar{\mu}_{X_{train}},
				\bar{\Sigma}_{X_{train}})$
				with $\bar{\mu}_{train} = f_{train}$ and $\bar{\Sigma}_{train} = {\displaystyle O}$
			\end{itemize}

\section{Blood Pressure Time Series Simulation}\label{sec:blood-pressure-time-series-simulation}

In section \ref{sec:characteristics-of-the-blood-pressure-time-series} listed
the properties the simulated BP time series process and the measurements should have.

Recall that we assume the following model for the BP measurements $Y(x)$ at a time point $x$, as
described in section \ref{sec:characteristics-of-the-blood-pressure-time-series}:

\begin{align*}
    Y(x) = f(x) + \epsilon && \epsilon \sim \N(0, \sigma_n^{2})
\end{align*}

The BP measurements $Y(x)$ are obtained by adding iid Gaussian measurement noise
$\epsilon$ to the BP time series process $f(x)$, which is assumed
to be a random function.

The time series process $f(x)$ was decided to be modeled by a Gaussian process.

The following Gaussian process was chosen to match the properties from
section \ref{sec:characteristics-of-the-blood-pressure-time-series}:

\begin{gather*}
    f(x) \sim GP(120, k(x,x'))
\end{gather*}
with:
\begin{align*}
    k(x, x') \text{ } = &\text{ } 2.24^{2} * \text{Matérn(l=3, $\nu$=0.5)} \\
             &+  14^{2} * \text{Periodic(l=3, p=24)} \\
             &+  2.24^{2} * \text{RBF(l=50)}
\end{align*}
where l denotes the length scale and p denotes the periodicity
of the corresponding kernel function in hours.
The formal definition of the Matérn, Periodic and RBF kernel
functions and their parameters is provided in section \ref{sec:kernel}.
Each of these kernels models one of the components described in
\ref{subsec:blood-pressure-time-series-process}:
\begin{itemize}
    \item The Matérn kernel with $\nu$=0.5 models the AR(1) component
    \item The Periodic kernel models the circadian cycle
    \item The RBF kernel model a long term trend
\end{itemize}






\section{Gaussian Process Regression}\label{sec:gaussian-process-regression}


\section{Baseline Methods}\label{sec:baseline-methods}


\section{Target Measures}\label{sec:target-measures}

\subsection{Time in Target Range}




\section{Impact of the Sampling Pattern}\label{sec:adversarial-analysis}


\section{Performance Metrics}\label{sec:performance-metric}




















