\chapter{Methods}\label{ch:methods}

The last chapter introduced Gaussian process regression
to establish a mapping between a time point x and its
corresponding BP value.
Since Gaussian Processes are capable of modeling time series in continuous
time and hence deal with irregularly spaced data,
they seem to be a good candidate for modeling a time series from which
we only have irregularly sampled noisy measurement.
To identify how suited GPs are for the problem at hand, the plan is to:
\begin{itemize}
    \item Simulate a BP time series and some noisy measurements of the time series
    \item Estimate certain time series characteristics (target measures) from these measurements.
    \item Compare the aptness of GP regression to estimate the time target measures to that of some baseline methods.
    \item Investigate the impact of adversarial factor on the target measure estimates, e.g
    How densely does the data need to be sampled to get good estimates?
    What happens if the data is not uniformly sampled across time ?
\end{itemize}

The next section will provide you with an overview of the steps involved
in quantifying the performance of a given method.


\section{Overview}

\begin{enumerate}
    \item Simulate BP time series at high resolution (section \ref{sec:blood-pressure-time-series-simulation})
    \item Subsample data (section \ref{sec:adversarial-analysis})
    \item Fit Regression (For GP Regression see section \ref{sec:gaussian-process-regression}
    for baseline methods see \ref{sec:baseline-methods})
    \item Extract target measure including CI (section \ref{sec:target-measures})
    \item Calculate performance metric between true and predicted values (section \ref{sec:performance-metric})

\end{enumerate}




	\begin{itemize}
		\item Simulate BP Time Series
	\end{itemize}

	\bigskip % Vertical whitespace

	\begin{itemize}
		\item For target measure compare performance of GP regression to baseline methods:
		\begin{itemize}
			\item Target measures: weekly daily and hourly mean, TTR
			\item Baseline Methods: Linear Regression, empirical overall mean, ttr naive, cubic spline
		\end{itemize}
	\end{itemize}

	\bigskip % Vertical whitespace

	\begin{itemize}
		\item Adversarial analysis: compare performance of baseline and GP regression introducing different adversarial
		factors
		\begin{itemize}
			\item Model mis-specification (mis-specified kernel or hyperparameters and non-gausianities)
			\item Missing data. More or less data available, more or less extreme non-uniform sampling patterns
		\end{itemize}

	\end{itemize}




\begin{algorithm} \caption{Simulation and Evaluation Flow}
 \hspace*{\algorithmicindent} \textbf{Input:} $t$, SamplingScheme, $GP_{true}$,
RegressionMethod, TargetMeasure\\
 \hspace*{\algorithmicindent} \textbf{Output:} CiCoverage, CiWidth
\begin{algorithmic}[1]
    \For {$i \gets 1$ to $N$}

        \State $f \gets$ Sample from $GP_{true}$
        \State $y \gets f + \epsilon$, $\epsilon_i \text{ i.i.d.} \sim \N(0, \sigma_n^2)$
        \State $y_{train} \gets$ subsample $y$ based on sampling scheme
        \State RegressionMethod.fit($y_{train} $)
        \State $\hat{y} \gets$ RegressionMethod.predict($t$)
        \State ci coverage $\gets$ vla
        \EndFor
    \Ensure $V \approx V^\pi$
\end{algorithmic}
\end{algorithm}


\section{Simulation and Evaluation Flow GP}

Simulation of true BP signal and measurements:
    \begin{itemize}
        \item $X=\{x_1, \dots x_n\}$: The input time points of interest, i.e. 1 week with 10 BP values per hour.
        \item $f_X := \{f(x_1) \dots f(x_n)\}$: The true BP signal $f(x)$ drawn from $GP(0, k_{true}(x,x'))$ evaluated at inputs $X$.
        \item $y_X := \{y(x_1) \dots y(x_n)\}$: The noisy BP measurements. $y(x)= f(x) + \epsilon$ with $\epsilon \sim N(0, \sigma_n^2)$
    \end{itemize}

Based on subsampling scheme choose $X_{train} \subset X$ with $|X_{train}| = m$ and $X_{test} = X \setminus X_{train}$.
\begin{itemize}
    \item $y_{train} := \{y(x_i) | x_i \in X_{train}\}$
    \item  $f_{train} := \{f(x_i) | x_i \in X_{train}\}$
\end{itemize}


	Fit GP regression model to training data $y_{train}$, $X_{train}$:
    \begin{itemize}
        \item $k_{fit}$: The fitted kernel function with hyperparameters $\theta$ that maximizes the marginal likelihood
        $p(y_{train}| \theta)$
        \item $p(f_X| y_{train}, k_{fit}) = N(\bar{\mu}, \bar{\Sigma})$:
        The posterior (predictive) probability density of $f_X$. The posterior mean vector
        $\bar{\mu} \in \mathbb{R}^n$ contains the point estimates for $f_X$.
        \begin{itemize}
            \item $\bar{\mu}_{train} := \{\bar{\mu}_i | x_i \in X_{train}\}$
            \item $\bar{\Sigma}_{train} := \bar{\Sigma}_{i,j}$ where $\{i | x_i \in X_{train}\}$ and $\{j | x_j \in X_{train}\}$

        \end{itemize}
    \end{itemize}

	Output of GP Regression are predictive probabilities $p(f_{train}| y_{train})$,
		$p(f_{test}| y_{train})$ and $p(f_{X}| y_{train})$:
			\begin{itemize}
				\item Note $\bar{\mu}_{train} \neq f_{train}$ due to the measurement error. \\ If
				$f_{train}$ was known: $p(f_{train}| f_{train}, GP_{fit}) = N(\bar{\mu}_{X_{train}},
				\bar{\Sigma}_{X_{train}})$
				with $\bar{\mu}_{train} = f_{train}$ and $\bar{\Sigma}_{train} = {\displaystyle O}$
			\end{itemize}

\section{Blood Pressure Time Series Simulation}\label{sec:blood-pressure-time-series-simulation}



\section{Gaussian Process Regression}\label{sec:gaussian-process-regression}


\section{Baseline Methods}\label{sec:baseline-methods}


\section{Target Measures}\label{sec:target-measures}


\section{Adversarial Analysis}\label{sec:adversarial-analysis}


\section{Performance Metrics}\label{sec:performance-metric}




















