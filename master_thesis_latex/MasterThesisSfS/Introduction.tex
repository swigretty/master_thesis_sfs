\chapter{Introduction}\label{ch:introduction}

\section{Motivation and Thesis Objective}\label{sec:thesis-objective}

This thesis aims at presenting Gaussian process regression as a powerful tool for
modeling time series based on irregularly spaced observations.

The motivation for this research stems from a pressing real-world problem in the
field of medicine, which will serve as a recurring example throughout this thesis. 
The problem revolves around estimating critical time series properties,
from a dataset consisting of irregularly spaced blood pressure (BP) measurements.
High BP is a well-established risk factor for cardiovascular disease, and
summarizing an individual's BP levels typically involves calculating the average
BP value over available measurements within a specified time range. A novel
monitoring device has been developed by the company Aktiia.
The device collects continuous BP estimates by
converting photoplethysmography (PPG) signals into BP measurements. 
The sampling frequency of this system can vary widely, typically yielding around
1.5 BP measurements per hour. 
However, factors such as PPG signal quality and external conditions can 
influence this frequency, resulting in irregularly spaced
measurements.
Obtaining accurate estimates of true BP values at unobserved time points is
essential for improving cardiovascular risk assessment and developing valuable
metrics.

Standard time series analysis methods traditionally assume discrete equispaced
time intervals. Introductory textbooks on time series analysis either neglect
the irregularly spaced case entirely or dedicate only a limited section to
continuous time models or state-space models with missing observations (\citeauthor{brockwell_time_1991}, \citeauthor{brockwell_introduction_2016},
\citeauthor{cryer_time_2008}, \citeauthor{chatfield_analysis_2003}).

Therefore, the primary objective of this thesis is to address the challenges
posed by irregularly sampled time series data and demonstrate why conventional
time series methods fail to deal with it. Additionally, we will elucidate why Gaussian
processes are a suitable approach for modeling time series with irregularly
spaced observations, using the BP time series example.

\section{Problem Statement}\label{sec:problem-statement}

To begin modeling BP measurements, we introduce the time series process $Y(x)$,
which combines the true BP process $f(x)$ with independent and identically
distributed (iid) Gaussian measurement noise $\epsilon$:

\begin{align*}
    Y(x) = f(x) + \epsilon && \epsilon \sim \N(0, \sigma_n^{2})
\end{align*}

Both time series, $f(x)$ and $Y(x)$, are described as random functions.
While the former is completely unobserved we have
unequally spaced observations from the latter, i.e. $(Y_{t_i}: i \in \{1, 2, \dots n\})$.
These observations represent Aktiia's user data.

The goal of this research is to learn about the underlying true BP process
$f(x)$ based on one week of irregularly spaced observations.
Instead of using real data, data will be simulated by generating the true BP
process $f(x)$ and adding measurement noise $\epsilon$.
This approach offers the advantage of
complete knowledge about $f(x),$ enabling us to quantify the accuracy of its
reconstruction from data. However, it also introduces the challenge of simulating
a time series and observations that closely mimic reality.
The time series characteristics to mimic are desribed in the next subsection
\ref{sec:characteristics-of-the-blood-pressure-time-series}.

Instead of solely focusing on predicting $f(x),$ this thesis emphasizes a set of
target measures deemed most relevant for assessing cardiovascular risk. These
target measures are detailed in subsection \ref{subsec:target-measures}.

In addition to point estimates, this research considers the construction of
confidence intervals (CIs) around these estimates. Notably, the width of the CI
intervals around the mean function varies over time, depending on factors such as
the availability of data in the vicinity of a given time point.

For simplicity, this study exclusively deals with systolic blood pressure and does
not consider diastolic measurements. All references to "blood pressure" or "BP"
pertain to systolic blood pressure.


\subsection{Characteristics of the Blood Pressure Time Series}\label{sec:characteristics-of-the-blood-pressure-time-series}

Based on the Aktiia user data, several properties of \textbf{the BP
measurements, $(Y_{t_i}: i \in \{1, 2, \dots n\})$}, have been identified:

\begin{enumerate}[\roman{enumi}.)]
\item The measurements are irregularly spaced, meaning that the time between
consecutive measurements varies.
\item Observations are not uniformly sampled across time; instead, their
density follows a circadian cycle, resulting in seasonal sampling.
\item The sampling frequency ranges from 0.5 to 4 measurements per hour.
\item The difference between average daytime and nighttime BP measurements
falls within the range of 0 to 20 mmHg, with an average difference of 10 mmHg.
\item The mean BP across all users is 120 mmHg.
\item The within-subject one-week sample variance spans from 16 to 144
mmHg², with an average of 49 mmHg².
\end{enumerate}


\textbf{The true BP time series process, $f(x)$}, cannot be directly observed.
However, in this thesis, it
is assumed to be a combination of the following components:

\begin{itemize}
\item A seasonal component representing the circadian cycle, as BP tends to be
higher during the day than at night.
\item An autoregressive component, reflecting the dependence of the output
variable on its previous values.
\item A long-term trend.
\end{itemize}

The magnitude of the measurement noise, denoted as $\epsilon,$ remains unknown.
Nevertheless, Aktiia measurements have undergone validation against a reference
method. The measured variance of the differences between Aktiia measurements and
this reference is 62 mmHg². Consequently, we can express:

\begin{align*}
    \Var(BP_{Ref} - BP_{Aktiia})
    & = \text{62 mmHg\textsuperscript{2}} = \Var(\epsilon_{Ref} - \epsilon_{Aktiia}) \\
    & = \Var(\epsilon_{Ref}) + \Var(\epsilon_{Aktiia}) - 2\Cov(\epsilon_{Ref},
    \epsilon_{Aktiia})
\end{align*}

Assuming that the noise variance of the reference method, $\Var(\epsilon_{Ref}),$
equals that of the Aktiia measurements, $\Var(\epsilon_{Aktiia}),$ and that
$\Cov(\epsilon_{Ref}, \epsilon_{Aktiia})=0,$ we would obtain a noise variance for the Aktiia
measurements of 31 mmHg².

\subsection{Target Measures}\label{subsec:target-measures}

The primary focus of this research lies on a set of target measures crucial for
estimating an individual's cardiovascular risk. These measures include:

\textbf{The mean BP} calculated over different time windows, such as one-hour,
one-day, and one-week mean BP. The mean BP is a pivotal and frequently reported
metric. Presently, it is computed based on the available measurements within the
corresponding time range.

\textbf{Time in Target Range (TTR)} evaluates the duration during which BP values
fall within a specified target range relative to the total time. It is currently
determined by dividing the number of BP measurements within the range of 90 to
125 mmHg ("target range") by the total number of BP measurements available
within one week.

It is noteworthy that the estimation of these target measures does not depend on
forecasting future BP values but solely relies on predicting BP values within the
one-week range of available data. Consequently, this thesis concentrates on
reconstructing BP values between the first and last time point in the dataset.

\section{Thesis Outline}

[Insert your thesis outline here]