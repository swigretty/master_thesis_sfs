\chapter{Introduction}\label{ch:introduction}

\section{Motivation and Thesis Objective}\label{sec:thesis-objective}

The thesis aims at presenting Gaussian process regression for modeling time
series based on irregularly spaced observations.

The topic is motivated by a "real world" problem from medicine which will
serve as a running example throughout the thesis.
%
%Motivated by a "real world" problem from medicine, I have tried to identify the most important concepts and basic
%methods for the analysis of irregularly space time series, which I will be presented in this thesis.
The problem at hand is the one of estimating certain time series properties (target measures) from a dataset
featuring systolic blood pressure (BP) measurements sampled at irregularly spaced time points.
High BP is known to be a risk factor for cardiovascular disease.
A person’s BP level is generally summarized using the average BP value over available measurements within a given time range.
A novel monitoring device already allows to collect BP estimates round the clock.
The device is collecting photoplethysmography (PPG) signals and converting them into BP measurements.
Typically, the system will yield approximately 1.5 BP measurements per hour, but depending on the quality of the PPG signal and some additional external factors,
this sampling frequency can widely vary and the expected range lies roughly between 0 and 5 measurements per hour.
Having good estimates of the true BP values at any, potentially not observed, time would allow for a better estimation
of the person’s cardiovascular risk, and enable the development of novel valuable metrics.

The standard time series analysis methods usually assume discrete equispaced
time and introductory textbooks on time series analysis either completely omit
the irregularly spaced case or they only dedicate
a very small section to continuous time models or to state-space models
with missing observations (\citeauthor{brockwell_time_1991}, \citeauthor{brockwell_introduction_2016},
\citeauthor{cryer_time_2008}, \citeauthor{chatfield_analysis_2003}).
The aim is thus to understand the problem of irregularly sampled time series data
and why standard time series method cannot deal with it.
Furthermore, the BP time series example should elucidate why and when Gaussian processes are a
suitable method to model a time series from which we only have observations sampled at
irregularly spaced time points.


\section{Problem Statement}

Let us start by modelling the BP measurements with a time series process, that we call
$Y(x)$ and that is a combination of the true BP process $f(x)$ and some iid
Gaussian measurement noise $\epsilon$:

\begin{align*}
    Y(x) = f(x) + \epsilon && \epsilon \sim \N(0, \sigma_n^{2})
\end{align*}

Note that both time series $f(x)$ and $Y(x)$ are described as random functions,
whereas the former is completely unobserved and from the latter we have
unequally spaced observations $(Y_{t_i}: i \in \{1, 2, \dots n\})$.
These observations represent AAAA's user data.

The goal is to learn something about $f(x)$ based on one week of such unequally
spaced observations.
Instead of using real data, data will be simulated through simulation of the
true BP process, $f(x)$, and by adding measurement noise, $\epsilon$.
This way we have complete knowledge about $f(x)$, allowing us to quantify
how well it was reconstructed from data.
However, this approach also imposes the additional challenge
of simulating a time series and observations that mimic reality as close
as possible.
The time series characteristics to mimic are desribed in the next subsection
\ref{sec:characteristics-of-the-blood-pressure-time-series}.

Instead of assessing the performance of predicting $f(x)$,
the thesis will focus on a set of target measures,
which have been considered most relevant for estimating the person’s cardiovascular risk.
These target measure are desribed in subsection \ref{subsec:target-measures}.

Besides the point estimates also their CIs are of interest.
Importantly, the CI should be able to capture the uncertainty due to the lack
of data in the proximity of the point of prediction.
This implies, that the width of the CI intervals around the mean function will
not be constant over time but depend, among
other factors, on how much data is available in the proximity of a given time point.

Note, for simplicity reasons, we only consider systolic and never diastolic
blood pressure.
Whenever blood pressure or BP is mentioned, it refers to systolic blood pressure.


\subsection{Characteristics of the Blood Pressure Time Series}\label{sec:characteristics-of-the-blood-pressure-time-series}

Based on the AAAA user data, the following properties of \textbf{the BP
measurements, $(Y_{t_i}: i \in \{1, 2, \dots n\})$}, have been identified:

\begin{enumerate}[\roman{enumi}.)]
    \item Measurements are irregularly spaced, i.e., the time between two
    consecutive measurements varies.
    \item Measurements are not uniformly sampled across time, but the density of
    the observations should follow the circadian cycle (seasonal sampling).
    \item The sampling frequency varies from 0.5 to 4 measurements per hour.
    \item The difference between the average day and average night BP measurements
    is between 0 and 20 mmHg and is on average 10 mmHg.
    \item The mean BP overall users is 120 mmHg.
    \item The within-subject one-week sample variance is between 16 and 144
    mmHg\textsuperscript{2} and is on average 49 mmHg\textsuperscript{2}.
\end{enumerate}


\textbf{The true BP time series process, $f(x)$}, cannot be directly observed,
however in the context of this thesis, it
is assumed to be a combination of the following components:

\begin{itemize}
    \item A seasonal component, representing the circadian cycle. BP is known to
    be higher during the day than during the night.
    \item An autoregressive component, since we assume the output variable to
    depend on its own previous values.
    \item A long-term trend.
\end{itemize}


The magnitude of the \textbf{measurement noise}, $\epsilon$, is also unknown.
However, the AAAA measurements have been validated against some
reference method.
The measured variance of the differences between AAAA measurements
and this reference is 62 mmHg\textsuperscript{2}.
We can thus write:
\begin{align*}
    \Var(BP_{Ref} - BP_{AAAA})
    & = \text{62 mmHg\textsuperscript{2}} = \Var(\epsilon_{Ref} - \epsilon_{AAAA}) \\
    & = \Var(\epsilon_{Ref}) + \Var(\epsilon_{AAAA}) - 2\Cov(\epsilon_{Ref},
    \epsilon_{AAAA})
\end{align*}

If one further assumes that the noise variance of the reference method,
$\Var(\epsilon_{Ref})$, equals that of the AAAA measurements, $\Var(\epsilon_{AAAA})$,
and that $\Cov(\epsilon_{Ref}, \epsilon_{AAAA})=0$, we obtain a noise variance
for the AAAA measurements,
$\Var(\epsilon_{AAAA})$, of 31 mmHg\textsuperscript{2}.


\subsection{Target Measures}\label{subsec:target-measures}
The set of target measures,
which have been considered most relevant for estimating the person’s cardiovascular risk are
the mean BP over different time windows and time in target range:

\textbf{The mean BP} should be calculated over different time-windows,
i.e. one-hour, one-day and one-week mean BP.
The mean is the most relevant and most often reported quantity.
Currently the mean is calculated based on the available measurements
within the corresponding time range.

\textbf{Time in Target Range (TTR)} assesses the time period where the
BP values are within a specified target range, relative to the total time.
It is currently calculated by dividing the number of BP measurements which are
within the range of 90 to 125 mmHg by the total number of BP measurements
available within one week.

Note that the estimation of these target measures does not depend on
forecasted future BP values but only on predicted BP values within the
one-week range of available data.
Hence, the thesis will only focus on the task of reconstructing BP values
between the first and last time point in the dataset.


\section{Thesis Outline}










%%% Local Variables: 
%%% mode: latex
%%% TeX-master: "MasterThesisSfS"
%%% End: 
