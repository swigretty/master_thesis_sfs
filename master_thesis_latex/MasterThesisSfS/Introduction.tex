%! Author = gianna
%! Date = 05.04.23
\chapter{Introduction}\label{ch:introduction}


\section{Thesis Objective}\label{sec:thesis-objective}

The thesis aims at presenting Gaussian process regression as
a time series analysis method for irregularly sampled data.

The topic is motivated by a "real world" problem from medicine which will
serve as a running example throughout the thesis.
%
%Motivated by a "real world" problem from medicine, I have tried to identify the most important concepts and basic
%methods for the analysis of irregularly space time series, which I will be presented in this thesis.
The problem at hand is the one of extracting time series characteristics from a dataset
featuring blood pressure (BP) measurements sampled at irregularly spaced time points.
High BP is known to be a risk factor for cardiovascular disease.
A person’s BP level is generally summarized using the average BP value over available measurements within a given time range.
A novel monitoring device already allows to collect BP estimates round the clock.
The device is collecting photoplethysmography (PPG) signals and converting them into BP measurements.
Typically, the system will yield approximately 1.5 BP measurements per hour, but depending on the quality of the PPG signal and some additional external factors,
this sampling frequency can widely vary and the expected range lies roughly between 0 and 5 measurements per hour.
Having good estimates of the true BP values at any, potentially not observed, time would allow for a better estimation
of the person’s cardiovascular risk, and enable the development of novel valuable metrics.
The thesis will focus on a set of time series characteristics (target measures),
which have been considered most relevant for estimating the person’s cardiovascular risk.
The target measures are:
\begin{itemize}
    \item the mean function of the BP time series
    \item the one-hour, one-day and one-week mean BP value
    \item Time in target range
    \item characteristics of the circadian cycle, such as the mean day and night BP
\end{itemize}
Besides the point estimates also their CIs are of interest.
Importantly, the CI should be able to capture the uncertainty due to the lack of data in the proximity of the point of prediction.
This implies, that the width of the CI intervals around the mean function will not be constant over time but depend, among
other factors, on how much data is available in the proximity of a given time point.
The described endpoints are all based on prediction at the not observed passed time points however not on forcasting at new time points in
the future.
Hence, the thesis will only focus on the task of reconstructing BP values between the first and last time point in the dataset.

The standard time series analysis methods usually assume discrete equispaced
time and introductory textbooks on time series analysis either completely omit
the irregularly spaced case or they only dedicate
a very small section to continuous time models or to state-space models
with missing observations (\citeauthor{brockwell_time_1991}, \citeauthor{brockwell_introduction_2016},
\citeauthor{cryer_time_2008}, \citeauthor{chatfield_analysis_2003}).
The aim is thus to understand the problem of irregularly sampled time series data
and why standard time series method cannot deal with it.
Furthermore, the BP time series example should elucidate why and when Gaussian processes are a
suitable method to model a time series from which we only have observations sampled at
irregularly spaced time points.
Although the topic is motivated by a real dataset we will restrict ourselves to simulated data,
which will mimic the most important characteristics of BP time series data.

\section{Characteristics of the Blood Pressure Time Series}\label{sec:characteristics-of-the-blood-pressure-time-series}
TODO
circadian cycle



\section{Thesis Outline}

The aim is thus to understand the problem of irregularly sampled time series data
and why standard time series method cannot deal with it.
Furthermore, it should be elucidated why and when Gaussian processes are a
suitable method to model a time series from which we only have observations sampled at
irregularly spaced time points.










%%% Local Variables: 
%%% mode: latex
%%% TeX-master: "MasterThesisSfS"
%%% End: 
