\chapter{Complementary information}\label{app:complement}


%Additional material. For example long mathematical derivations could be
%given in the appendix. Or you could include part of your code that is
%needed in printed form. You can add several Appendices to your thesis (as
%you can include several chapters in the main part of your work).
%
%\section{Ornstein-Uhlenbeck Process}\label{app:ou}
%
%The autocovariance function of an Ornstein-Uhlenbeck process can be derived by solving the stochastic differential equation (SDE) that defines the process.
%
%Starting with the SDE for an OU process:
%
%$$dX_t = \theta (\mu - X_t)dt + \sigma_w dW_t,$$
%
%where $X_t$ is the value of the process at time $t$, $\theta$ is a positive constant that determines the speed of mean reversion,
%$\mu$ is the long-term mean of the process, $\sigma_w$ is the standard deviation of the random shocks, and $W_t$ is a standard Wiener process or Brownian motion.
%
%The solution to the SDE is:
%
%$$ X_t = X_0 e^{-\theta t} + \mu (1-e^{-\theta t}) +
%\sigma_w e^{-\theta t} \int_{0}^{t} e^{\theta s} dW_s$$
%
%The process is stationary if $\theta > 0$.
%The autocovariance function of an OU process is given by
%$Cov(X_t, X_{t-k}) = \frac{\sigma_w^2}{2\theta} e^{-\theta k}$,
%where $k\geq 0$ and $\theta > 0$.
%
%This is the same expression as we have obtained in \ref{kernel-matern-ar1}, where
%$k(0) = \sigma^2 = \frac{\sigma_w^2}{2\theta}$ and $l=1/\theta$
%
%To see how the Ornstein-Uhlenbeck can be considered a continuous time analogue to the discrete time
%AR(1) process one can use the Euler-Maryuama discretization of the process.
%Considering again the SDE for an OU process:
%$$dX_t = \theta (\mu - X_t)dt + \sigma_w dW_t,$$
%The process can be discretized at times $(k \Delta t)_{k \in \mathbb{N}_0}$:
%
%$$ X_{k+1} - X_k = \theta \mu \delta t - \theta X_k \Delta t + \sigma_w (W_{k+1} - W_k)$$
%
%The random variables $(W_{k+1} - W_k)$ are independent and identically distributed normal random variables
%with expected value zero and variance $\Delta t$.
%Therefore, we can set $\sigma_w (W_{k+1} - W_k) = \sigma_w \sqrt{\Delta t} \epsilon$ with $\epsilon \sim \N(0,1)$
%to obtain the following recursion:
%$$ X_{k+1} = \theta \mu \Delta t - (\theta \Delta t - 1) X_k + \sigma_w \sqrt{\Delta t} \epsilon$$
%
%The recursion for an AR(1) process is:
%$$ X_{k+1} = c + a X_k + b \epsilon$$
%Which is identical to the expression above if $c= \theta \mu \Delta t$, $a=1- \theta \Delta t$ and
%$b= \sigma_w \sqrt{\Delta t}$
%


\section{Properties of the Simulated BP Measurements}\label{sec:properties-of-the-simulated-time-series-samples}

This section presents the distribution of some crucial properties
of the simulated BP measurements.
These histograms have been created by drawing 100 samples from the true GP and
adding iid Gaussian measurement noise.

The goal of the simulation was to mimic the real-world BP time series
characteristics described in section \ref{sec:characteristics-of-the-blood-pressure-time-series}.

\begin{figure}[h!]
    \centering
    \includegraphics[width=0.6\linewidth]{
        Pictures/variance_distribtution/sin_rbf_seasonal_default/09_07_07_12_53/variance_y_true_train_summary.pdf}
    \caption[Distribution of One-Week BP Variance from Simulated Measurements]{
        Distribution of One-Week BP Variance from Simulated Measurements:
    The one-week variance should span from from 16 to 144 mmHg², with an average of 49 mmHg²}
    \label{fig:variance}
\end{figure}

\begin{figure}[h!]
    \centering
    \includegraphics[width=0.6\linewidth]{
       Pictures/variance_distribtution/sin_rbf_seasonal_default/09_07_07_12_53/variance_dip_ampl_summary}
    \caption[Distribution of the Night Dip Magnitude from Simulated Measurements]{
        Distribution of the Night Dip Magnitude from Simulated Measurements:
    Here the night dip magintude is defined as half of the difference between average daytime and nighttime BP measurements.
        Should fall within the range of 0 to 10 mmHg}
    \label{fig:dip_ampl}
\end{figure}



\begin{figure}[h!]
    \centering
    \includegraphics[width=0.6\linewidth]{
       Pictures/variance_distribtution/sin_rbf_seasonal_default/09_07_07_12_53/variance_2.24**2 * Matern(length_scale=3, nu=0.5) * 1**2_summary.pdf}
    \caption[Distribution of the AR Component Variance from Simulated Measurements]{
    Distribution of the AR Component Variance from Simulated Measurements:
    There exists no target values for the variance of the AR component.}
    \label{fig:var_matern}
\end{figure}

\begin{figure}[h!]
    \centering
    \includegraphics[width=0.6\linewidth]{
       Pictures/variance_distribtution/sin_rbf_seasonal_default/09_07_07_12_53/variance_2.24**2 * RBF(length_scale=50) * 1**2_summary.pdf}
    \caption[Distribution of the RBF Component Variance from Simulated Measurements]{
        Distribution of the RBF Component Variance from Simulated Measurements:
        There exists no target values for the variance of teh RBF components}
    \label{fig:var_rbf}
\end{figure}


\begin{figure}[h!]
    \centering
    \includegraphics[width=0.6\linewidth]{
       Pictures/variance_distribtution/sin_rbf_seasonal_default/09_07_07_12_53/variance_14**2 * ExpSineSquared(length_scale=3, periodicity=24) * 1**2_summary.pdf}
    \caption[Distribution of the Periodic Component Variance from Simulated Measurements]{
        Distribution of the Periodic Component Variance from Simulated Measurements:
        The target ranges for the variance of the periodic component are provided in the form of night dip values (see Figure \ref{fig:dip_ampl})}
    \label{fig:var_periodc}
\end{figure}

